The results suggest that the proportion of high achievers in a class has a significant, but small, negative impact on the grades of middle achievers, while the proportion of low achievers in a class has a significant, but small, positive impact on the grades of middle achievers\footnote{See \tablelabel{tab:2sc}, \tablelabel{tab:3sc}, and \tablelabel{tab:4sc}}.
We find that the estimated coefficients are too small to have an economically significant impact on final student grades, however if these trends continue in the long run they may impact cumulative GPA. 
These results are contradictory to those found in the majority of the literature and we discuss some possible explanations for this below \citep{kang2007classroom,carman2012classroom,burke2013classroom,schlosser2008inside,lavy2012good}. 
We begin the section by evaluating our results in detail and describing the patterns found.
Then, we conclude by discussing the implications of our results and provide some suggestions for further research. 

%It should be noted that even though we attempt to correct for any selection bias that may exist, our results should still be interpreted with caution, as there may exist model misspecifications that still make our model susceptible to selection bias. 

It is important to note that the main results of this paper focus on the impact of high achievers and low achievers on the grades of middle achievers. 
Unfortunately, data limitations made it difficult to measure the impact of classmate ability on high achievers or low achievers.\footnote{Specifically we could not measure the impact of the proportion of low achievers and middle achievers on high achievers and similarly for low achievers.} 
There are not enough, observations at this time, within either group (high achievers and low achievers) to run the two stage regression model. 
However, the regressions run on the middle achievers still make an important contribution as the majority of students are middle achievers, so our findings impact most of the population.

\tablelabel{tab:2sc}, \tablelabel{tab:3sc}, and \tablelabel{tab:4sc} show the relevant results\footnote{The results of the regression for all the variables can be found in \appendixlabel{appendix:b}.} of the two stage selection model (outlined in section~\ref{methods}) run on the middle achievers, where the selection equation \eqref{eq:1} is a two category ordered probit, three category ordered probit, and four category ordered probit respectively. 
Each table contains the relevant results from three regressions where the major difference between each regression within a table is the measure of the ``peer'' used. 
For the regressions found in \tablelabel{tab:2sc}, \ref{tab:3sc}, and \ref{tab:4sc}, we use the proportion of high achievers and low achievers in a class as a peer measure and regress on middle achieving students. 
After each regression, the measures used for a high and low achieving peer become stricter (and therefore the measure for the middle achievers becomes less strict). 
For instance, regression 1 (labeled Top/Bottom 25\%) in \tablelabel{tab:2sc} focuses on the impact of the proportion of high achievers, defined as the twenty-five percent with the highest academic rating in the sample, and the proportion of low achievers, the twenty-five percent with the lowest academic rating in the sample, on the grades of the middle achievers (those that are not high or low achievers). 
In each regression, GradeX refers to the regression run on the middle achievers that selected into their X choice class based on the first stage ordered probit.\footnote{For more details see \sectlabel{methods:tssmodel}} 

\clearpage{}

\begin{table}[htb]
  \centering
  \begin{threeparttable}
    \caption{Two Selection Categories}\label{tab:2sc}
    \begin{tabular}{l l l l} 
      \hline
      \hline
               & Top/Bott{\textasciitilde}25\% & Top/Bott{\textasciitilde}10\% & Top/Bott{\textasciitilde}5\% \\
               & (Std. Err.)                   & (Std. Err.)                   & (Std. Err.)                  \\
      \hline
      Grade1   &                               &                               &                              \\
      PctBotQ  & -0.150                        &                               &                              \\
               & (0.53)                        &                               &                              \\
      PctTopQ  & -0.787**                      &                               &                              \\
               & (0.41)                        &                               &                              \\
      PctBot10 &                               & 0.612                         &                              \\
               &                               & (0.54)                        &                              \\
      PctTop10 &                               & -0.741***                     &                              \\
               &                               & (0.33)                        &                              \\
      PctBot5  &                               &                               & 1.001**                      \\
               &                               &                               & (0.60)                       \\
      PctTop5  &                               &                               & -0.702**                     \\
               &                               &                               & (0.41)                       \\
      \hline
      Grade2   &                               &                               &                              \\
      PctBotQ  & 1.720**                       &                               &                              \\
               & (1.00)                        &                               &                              \\
      PctTopQ  & 0.660                         &                               &                              \\
               & (0.70)                        &                               &                              \\
      PctBot10 &                               & 0.370                         &                              \\
               &                               & (0.78)                        &                              \\
      PctTop10 &                               & 0.827**                       &                              \\
               &                               & (0.49)                        &                              \\
      PctBot5  &                               &                               & -0.061                       \\
               &                               &                               & (0.70)                       \\
      PctTop5  &                               &                               & 0.198                        \\
               &                               &                               & 0.198                        \\
      \hline
      \hline
    \end{tabular}
    \begin{tablenotes}
    \item{* p<.2, ** p<.1, *** p<.05 \\Note: Regressions run using 3 different peer measures and a two choice ordered probit for the first stage.}
    \end{tablenotes}
  \end{threeparttable}
\end{table}

\clearpage{}

%\thispagestyle{empty}
\begin{table}[htb]
  \centering
  \begin{threeparttable}
    \caption{Three Selection Categories}\label{tab:3sc}
    \begin{tabular}{l l l l} 
      \hline
      \hline
               & Top/Bott{\textasciitilde}25\% & Top/Bott{\textasciitilde}10\% & Top/Bott{\textasciitilde}5\% \\
               & (Std. Err.)                   & (Std. Err.)                   & (Std. Err.)                  \\
      \hline
      Grade1   &                               &                               &                              \\
      PctBotQ  & -0.314                        &                               &                              \\
               & (0.52)                        &                               &                              \\
      PctTopQ  & -0.840***                     &                               &                              \\
               & (0.39)                        &                               &                              \\
      PctBot10 &                               & 0.602                         &                              \\
               &                               & (0.54)                        &                              \\
      PctTop10 &                               & -0.705***                     &                              \\
               &                               & (0.33)                        &                              \\
      PctBot5  &                               &                               & 1.008**                      \\
               &                               &                               & (0.60)                       \\
      PctTop5  &                               &                               & -0.667*                      \\
               &                               &                               & (0.41)                       \\
      \hline
      Grade2   &                               &                               &                              \\
      PctBotQ  & 1.248                         &                               &                              \\
               & (1.41)                        &                               &                              \\
      PctTopQ  & 1.614**                       &                               &                              \\
               & (0.92)                        &                               &                              \\
      PctBot10 &                               & -0.138                        &                              \\
               &                               & (1.02)                        &                              \\
      PctTop10 &                               & 0.439                         &                              \\
               &                               & (0.70)                        &                              \\
      PctBot5  &                               &                               & -0.819                       \\
               &                               &                               & (0.88)                       \\
      PctTop5  &                               &                               & -0.585                       \\
               &                               &                               & (1.00)                       \\
      \hline
      Grade3   &                               &                               &                              \\
      PctBotQ  & 1.215                         &                               &                              \\
               & (1.54)                        &                               &                              \\
      PctTopQ  & -0.961                        &                               &                              \\
               & (1.16)                        &                               &                              \\
      PctBot10 &                               & 0.002                         &                              \\
               &                               & (1.18)                        &                              \\
      PctTop10 &                               & 0.541                         &                              \\
               &                               & (0.63)                        &                              \\
      PctBot5  &                               &                               & -0.170                       \\
               &                               &                               & (1.13)                       \\
      PctTop5  &                               &                               & -0.031                       \\
               &                               &                               & (0.83)                       \\
      \hline
      \hline
    \end{tabular}
    \begin{tablenotes}
    \item{* p<.2, ** p<.1, *** p<.05 \\Note: Regressions run using 3 different peer measures and a three choice ordered probit for the first stage.}
    \end{tablenotes}
    \centering
%    \thepage
  \end{threeparttable}
\end{table}

\clearpage{}

\begin{sidewaystable}[htb]
  \centering
  \begin{threeparttable}
    \caption{Four Selection Categories}\label{tab:4sc}
    \def\arraystretch{1.5}
    \begin{tabular}{l|c|c|c|c|c|c|c|c} 
      \hline
      \hline
      
      & \multicolumn{2}{|c|}{Grade1} & \multicolumn{2}{|c|}{Grade2} & \multicolumn{2}{|c|}{Grade3} & \multicolumn{2}{|c}{Grade4} \\
      \hline
      & \prbf{PctBot5} & \prbf{PctTop5} & \prbf{PctBot5} & \prbf{PctTop5} & \prbf{PctBot5} & \prbf{PctTop5} & \prbf{PctBot5} & \prbf{PctTop5} \\
      \hline
      Top/Bott{\textasciitilde}5\% & 0.998** & -0.669* & -0.791 & -0.530 & 2.443* & -0.215 & -0.417 & -2.847*** \\
      (Std. Err.) &(0.61) &(0.41) &(0.88) &(1.01) &(1.64) &(1.52) &(1.72) &(1.18) \\

      %         & \Top/Bott{\textasciitilde}5\% \\
      %         & (Std. Err.)                  \\
      % \hline
      % Grade1  &                              \\
      % PctBot5 & 0.998**                      \\
      %         & (0.61)                       \\
      % PctTop5 & -0.669*                      \\
      %         & (0.41)                       \\
      % \hline
      % Grade2  &                              \\
      % PctBot5 & -0.791                       \\
      %         & (0.88)                       \\
      % PctTop5 & -0.530                       \\
      %         & (1.01)                       \\
      % \hline
      % Grade3  &                              \\
      % PctBot5 & 2.443*                       \\
      %         & (1.64)                       \\
      % PctTop5 & -0.215                       \\
      %         & (1.52)                       \\
      % \hline
      % Grade4  &                              \\
      % PctBot5 & -0.417                       \\
      %         & (1.72)                       \\
      % PctTop5 & -2.847***                    \\
      %         & (1.18)                       \\
      \hline
      \hline
    \end{tabular}
    \begin{tablenotes}
    \item{* p<.2, ** p<.1, *** p<.05 \\Note: Regressions run using 3 different peer measures and a three choice ordered probit for the first stage. Additionally, Top/Bottom 25\% and Top/Bottom 10\% regressions could not be run because there is too little variance in the data at those levels (most likely because there are too few observations).}
    \end{tablenotes}
  \end{threeparttable}
\end{sidewaystable}

\clearpage{}

When analyzing the results in \tablelabel{tab:2sc}, \ref{tab:3sc}, and \ref{tab:4sc} there are a few important patterns to note. 
First, the majority of the statistically significant results are found in students who selected into their first choice (Grade1), and students who selected into their second choice or below (Grade2, Grade3, etc.) show inconsistent results. 
Next, all the significant results in Grade1 are consistent, in that the proportion of high achievers has a negative impact on the grades of middle achievers and the proportion of low achievers has a positive impact on the grades of middle achievers. 
Furthermore, the majority of significant results appear as the peer measure becomes stricter, that is more significant results appear as the peer measure changes from the top/bottom twenty-five percent, to ten percent, and finally to five percent. 
 
The Grade1 results in \tablelabel{tab:2sc}, \ref{tab:3sc}, and \ref{tab:4sc} show a consistent pattern. 
The proportion of high achievers in a classroom has a negative impact on the grade of middle achievers, regardless of the strictness of the peer measure. 
As the peer measure becomes stricter (specifically at the five percent threshold), we see that the proportion of low achievers in a classroom has a positive impact on the grades of middle achievers. 
There is some precedence for using the five percent threshold and finding peer effects. 
In \citet{lavy2012good} researchers also found strong peer effects using the top and bottom five percent as their peer measures. 
\citeauthor{lavy2012good} argue that their use of the top and bottom five percent is not arbitrary by showing that it is precisely those students around the five percent threshold that have a strong peer impact on fellow students. 
That is, they showed that students in the middle ninety percent of the ability distribution do not show strong peer effects of any sort, while the top and bottom five percent are the students who are the most influential. 

Our results run counter to those found in the current literature\footnote{\citet{burke2013classroom} do find one result similar to ours, but they do not provide any explaination for their finding.}, granted that the current literature is rather scarce especially in the area of higher education \citep{kang2007classroom,carman2012classroom,burke2013classroom,schlosser2008inside,lavy2012good}. 
Most of the current literature finds that low achievers hurt the grades of middle achieving students and high achievers help the grades of middle achieving students. 
Similar results have also been discovered in roommate peer effects literature \citep{griffith2014peer,zimmerman2003peer,sacerdote2000peer}.
The reasoning being that low achieving students may disrupt learning (possibly by instilling bad study habits in the middle achieving students) and the high achieving students may facilitate learning (possibly by asking more relevant questions or helping to tutor middle achieving students). 
This reasoning is very compelling and is exactly what many people would expect to happen, which makes our results of the contrary all the more puzzling.  

We have developed two possible explanations for these results. 
One possibility is that since most of the previous literature does not focus on courses in higher education, peer dynamics have changed in the transition from K-12 education to undergraduate education. 
Most of the literature focuses on K-12, therefore, it is reasonable to suppose that peer effects at institutions of higher education may differ from those seen in K-12 education. 
It is possible that high achieving students in college have a negative peer effect on grades, perhaps through demoralizing other students, and low achieving students have a positive peer effect on grades, perhaps through increasing the number of group study sessions. 
Another possibility is that grade curving is occurring in many of the classes, and is overpowering any actual peer effects. 
That is to say, a student's grade in a class may be determined by her relative performance to her classmates instead of by her absolute performance. 
Unfortunately, due to the time span of the data and faculty changes, it is not feasible to uncover which classes were truly graded on a curve (in which case these findings would not be surprising) and which classes were not. 
Furthermore, if grade curving is the underlying cause of the results we are finding, then it is still possible that peer effects that are in line with the literature exist, but the curving effects are simply overwhelming any peer effects that are occurring. 

Next, we'll explore the possible reasons behind the inconsistent results in the regressions run on students who selected into a class that was not their first choice. 
One reason, may be that for students who do not select into their first choice class, the peer effect dynamics change and aren't as strong. 
Perhaps the students who did not select into their first choice class have less motivation (or something of the sort) to perform in their non-first choice class, and this leads to less peer interactions (or interactions of a different kind) that ultimately lead to smaller/immeasurable peer effects. 
Another explanation, one that we believe is more likely, is that there are simply fewer students who selected into a course that is not their first choice and the results are therefore less reliable. 
In the best cases the number of students who did not select into their first choice is about one half of the number of students who did select into their first choice.\footnote{See \tablelabel{tab:freq_Ranking} in \sectlabel{data:pointsdata} for more details} 
That is, for a two option ordered probit selection model (used in \tablelabel{tab:2sc}), the number of students who are regressed in the Grade1 calculations (those who selected into their first choice class) is twice that of the number of students are regressed in the Grade2 calculations (those who did not select into their first choice class).\footnote{For more details see \sectlabel{methods}.} 
Additionally, as the number of options in the selection model increases (as is the case in \tablelabel{tab:3sc} and \ref{tab:4sc}) the number of students who did not select into their first choice class is further divided into those who selected into their second choice class, third choice class, etc. and regressed on. 
This essentially means that the regressions run on the students who did not select into their first choice class (Grade2, Grade3, Grade4) have far fewer observations than the regressions run on students who did select into their first choice class (Grade1). 
In this case, fewer observations effectively means a smaller sample size, and studies that use smaller sample sizes find that the sample size may lead to insignificant or inaccurate results \citep{gonzalez2013gibrat} and the results of these studies are often not generalizable \citep{oladipupo2013does}.
This is a troubling problem as it suggests that the results from the Grade2, Grade3, and Grade4 regressions are more unreliable than the results of the Grade1 regressions. 
However, the only way to resolve this problem is to collect several more years of data, which is not feasible at this time. 
Therefore, we will focus on the Grade1 regression results for the remainder of the section, and since the majority of students selected into their first choice class, they are arguably the most relevant results regardless. 

Additionally, our results are somewhat imbalanced. That is to say, high achieving students clearly have a negative impact on middle achieving students throughout the majority of our regressions, but low achieving students only have a statistically significant positive impact on middle achieving students when the peer measure becomes stricter. 
One explanation may be that it is pure statistical chance that high achievers had a significant negative affect more often than low achievers had a significant positive affect. 
Our sample size of 903 students who selected into their first choice class is not as large as some found in the literature, which reach into the many thousands \citep{kang2007classroom,lavy2012good}. 
Another possibility is that top students are very effective at having a negative impact on other student's grades, and this result is simply robust to changes in the peer measure. 
Finally, if grade curving is the underlying cause of our results, it may be that high achieving students dull out a middle achieving student's performance more than a low achieving student helps a middle achieving student shine.
Since professors may curve grades in a subjective fashion, they may be affected by this psychological phenomenon. 
Currently we do not have an explanation for this phenomena, although there is evidence of similar kinds of psychological imbalance in other areas. 
For example take the well documented theory of loss aversion. 
As explained by \citet{tversky1991loss}, loss aversion implies that people experience greater impact from a loss when compared to a gain equal in magnitude, and this phenomena has been found in many different areas of human behavior \citep{shalev2002loss,goette2004loss}. 
It is possible that the same psychological mechanisms driving the imbalance in losses and gains (loss aversion) are also driving the imbalance in the effect of the proportion of high achievers and low achievers on middle achievers grades. 

There are two concerns with the methodology used in this paper that need to be addressed.
One concern is selection bias and the other is the change in 2015 Demand\footnote{See \sectlabel{data:pointsdata} for more details.} which may affect the model and overall results.
In order to correct for selection bias we use the two stage selection model outlined in \sectlabel{methods:tssmodel}, and after comparing our results to the foundational OLS model outlined in \eqref{eq:0}, the overall results are the same.\footnote{See \appendixlabel{appendix:c} for the regular OLS results.} 
This suggests that selection bias is most likely not affecting our sample to any significant degree, but selection bias may still be a concern with any future studies that utilize a different sample.
To address our second concern, we correct Demand in 2015\footnote{See \sectlabel{data:pointsdata} for more details on the correction methods we used.} to eliminate some of the effects of the changes in the bidding system, however some effects persisted\footnote{In particular the standard deviation of Demand in 2015 was still affected. See \sectlabel{data} for more details.} and may have affected our results.
In order to verify the robustness of our results we reran the regressions on the same data from the years 2011-2014, with the original bidding system, and found that the same overall results persist.\footnote{TODO: See appendix X for the full results.}
Therefore neither of our initial concerns had a significant affect on the overall results of the paper.

\subsection{Discussion}\label{results:discussion}

The above results indicate the presence of an interesting trend at this institution, but are there any significant implications from these trends?
 In order to shed some light on the implications of our results, we must analyze the magnitudes of the estimated coefficients. 
The average class size in our data set is twelve students (which has remained fairly constant throughout the time period of our data set), and if we define high achievers as the top five percent of students then there is one student per class that is a high achiever,  assuming a uniform distribution of high achievers. 
At the five percent threshold, our results indicate that for every ten percent increase in the proportion of high achievers in the classroom the GPA of a middle achiever falls by roughly 0.07 points. 
In order for the proportion of high achievers in an average class to be high enough to have a significant\footnote{Defined as at least changing the sign of a student's grade which takes about 0.3 grade points.} influence on the grades of middle achieving students, forty percent of the average class (five students) must be high achievers.\footnote{0.4 * 0.7 = 0.28} Using the five percent threshold, there are two classes (about 1.5\%) in our entire data set which have a proportion of high achievers of at least forty percent. 
It is clear that having a class consist of at least forty percent high achievers is an unlikely scenario, and should not be a practical concern for most students. 
Similar calculations can be done to show that having a high enough number of low achieving students in a classroom to significantly affect grades is also quite uncommon. 
However, two considerations must be taken into account. 
First, if these results continued in further classes, then the cumulative effect on the GPA may be significant. 
Depending on a student's degree path, they may be more likely to select into classes with a higher proportion of low achieving students or classes with a higher proportion of high achieving students. 
Additionaly, higher level classes at the institution may number under ten students (depending on the popularity of the degree), therefore if these trends continue it is possible for the cumulative GPA (as well as class grades) of a student to be affected. 
This may be a serious problem in unpopular degrees with abnormally small class sizes. 
The second consideration is that the coefficients are estimates, and it may be that the real coefficients are much larger or smaller than those estimated. 
If the real coefficients happen to be larger, then the results may indeed be a practical concern to students and the administration.

The calculations above show that even though these trends exist, in the short run, the impact on final grades in the first course is most likely not a practical concern to most students. 
However, more research need to be conducted to determine the long run implications of these results. 

There are two types of policies that the majority of the literature discusses, ``streaming'' and ``mixing'' \citep{ding2007peers,kang2007classroom,carman2012classroom}. 
Streaming refers to the policy of grouping students of similar ability levels, while mixing generally refers to the policy of combining students of different ability levels \citep{ding2007peers}.
However, in this case it is difficult to suggest any policy implications without restricting student course choices and without knowing how high achievers and low achievers are affected by classmates of different ability levels. 

As \citet{burke2013classroom}, \citet{carman2012classroom}, and \citet{ding2007peers} find, an optimal policy prescription may not exist.
Suppose that the grades of high achievers benefit from middle and low achievers, while the grades of low achievers are harmed by middle and high achievers,\footnote{Which is what \appendixlabel{appendix:c} suggests to be the case.} then a trade off will need to be made if a policy is to be implemented. 
In this case, there is no grouping which would lead to a positive impact on all student grades. 
For instance, if middle achievers are prioritized, since there are more of them, then they should be placed into classes with as many low achievers as possible. 
However, this leads to two problems.
First, the grades of low achievers would be harmed in this case.
Additionally only a select group of middle achievers could benefit from this policy, since there are far fewer low achievers. 
Also, if too many middle achievers are placed into a class, the proportion of low achievers would be diluted to a point where the affects are insignificant. 
If, on the other hand, the grades of low achievers are positively impacted by the proportion of middle achievers in the class, then by grouping low achievers and middle achievers the grades of both may be maximized.
Again, only a select group of middle achievers would benefit, but this would still maximize the average grade received by all groups overall.
However, to effectively create this optimal grouping, the institution would need to decide to restrict student course choices based on their ability levels, which may not be a popular decision among current or prospective students.\footnote{Institutions could also incentivise students of certain ability levels to take certain courses, however this may be considered unfair to those students who are not incentivised.}

This study is somewhat limited by the sample size. 
With more observations we may be able to determine how high achievers and low achievers are affected by each other and middle achievers.
Additionally, it is an open question as to whether or not these results may be generalized to other types of institutions. 
This study was conducted on data from a small private liberal arts college and the results may differ at different types of institutions with potentially different peer interactions (such as at large universities). 
Also, due to the time span of the data, we are unable to determine whether peer effects, grade curving, or both are responsible for our results. 

Fortunately, these limitations may lead to promising areas of future research. 
In a few years there may be enough observations to rerun the regressions on high and low achievers and discover how they are affected by students of other ability levels in the class. 
With this full picture of classroom peer effects one might find unexpected peer effects (such as if low achievers benefited from high achievers and high achievers benefited from low achievers) that are currently unexplained and have clearer policy implications.
Another avenue for further research, would be to conduct similar regressions at different types of institutions.
It would be interesting to know if the same peer effects are found at community colleges or large universities. 
Both community colleges and large universities are a fundamentally different type of institution compared to a small private liberal arts school, and it is possible that different types (or the same types) of peer effects exist. 
Additionally, different liberal arts colleges may have different grading policies (i.e. a specific policy on grade curving), and by comparing the results from another (similar) institution to this one it may help determine if grade curving is an underlying cause of our results. 
Finally, it may be interesting to analyze classroom peer effects based on gender composition. 
\citet{oosterbeek2014gender} study gender peer effects in a university and find that ``males, but not females, perform poorer in courses with a high math component if the share of females in their work group increases.'' 
It is possible that the majority of our results are due to gender peer effects (instead of the ability peer effects that we attribute them to). 
That is, since high achievers are generally female (71\% are female at the 5\% threshold) the strongest driver of our overall results may be from the impact of a high achieving female on the grade of a middle achieving male. 

