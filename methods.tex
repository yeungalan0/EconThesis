Our foundational peer effects model is as follows:
\setlength{\belowdisplayskip}{6pt} \setlength{\belowdisplayshortskip}{1pt}
\setlength{\abovedisplayskip}{-4pt} \setlength{\abovedisplayshortskip}{1pt}

\begin{equation}\label{eq:0}
G_{i} = \beta_{0} + \beta_{1} Ability_{i} + \beta_{2} Ability_{i}^{CM1} + \beta_{3} Ability_{i}^{CM2} + \overrightarrow{\beta} \overrightarrow{z} + \epsilon_{i}
\end{equation}

This is an OLS model, where $G_{i}$ is the grade received by student $i$ in their first course at the institution, $Ability_{i}$ is a proxy for the student's academic ability (Academic Rating)\footnote{For more information see \tablelabel{tab:def1}.}, $Ability_{i}^{CM1}$ and $Ability_{i}^{CM2}$ are classmate ability measures, $\overrightarrow{z}$ is a vector of control variables\footnote{Control variables included Minority, Female, InState, International, Needy, Class Size, URM, Year, Division, and Professor. See \tablelabel{tab:def1} and \tablelabel{tab:def2} for definitions.}, and $\epsilon_{i}$ is the error term. 
$Ability_{i}^{CM1}$ and $Ability_{i}^{CM2}$ are our peer measures and are defined as one of the following, the proportion of high achieving, middle achieving, or low achieving students in a class\footnote{Unfortunately due to data limitations only the proportion of high achievers and low achievers in a class is used for our peer measures, for more details see \sectlabel{results}.}(as defined by cutoffs in academic rating), and $Ability_{i}^{CM1}$ is not the same measure as  $Ability_{i}^{CM2}$. 
Thus $\beta_{2}$ and $\beta_{3}$ are the primary coefficients of interest, as they estimate the impact of the classmates ability variables (our peer measures) on a student's grade. 
The OLS regression is run on the subset of students that are not part of the peer measures. %TODO: Fix wording part of...
For example, if $Ability_{i}^{CM1}$ and $Ability_{i}^{CM2}$ are defined as the proportion of high achievers in a class and the proportion of low achievers in a class respectively, then the model would be run on the middle achievers. 

However, because our sample is nonrandom\footnote{Unfortunately students are not assigned to classes and institutions randomly, instead the institution selects specific students, students next select the institution, and students then select into classes.} the calculated coefficients of model \eqref{eq:0} are at risk of being bias \citep{heckman1979sample}.\footnote{For more details see \sectlabel{methods:csb}} This bias, known as selection bias, is a serious threat to peer effects models of this type because self selection often exists and leads to a nonrandom sample \citep{carman2012classroom,burke2013classroom,ding2007peers}. 
Therefore, we must update our model to correct for any selection bias that may exist.

\subsection{Two Stage Selection Model}\label{methods:tssmodel}

As discussed in \sectlabel{litreview}, most studies of this kind exploit quasi-random student class assignments or utilize fixed effects models to control for selection bias \citep{kang2007classroom,carman2012classroom,schlosser2008inside,lavy2012good}. 
However, due to our unique access to student preference data, we use a two stage selection model to correct for selection bias, similar to the one described in \citet{heckman1979sample}. 
Our two stage selection model uses an ordered probit model in the first stage (the selection equation) and an OLS model in the second stage (the outcome equation). 
From the first stage ordered probit model we take the calculated inverse mills ratios and use them as a control variable in the second stage outcome equation. 
The inverse mills ratios are calculated estimates, that when used in the outcome equation, help to control for selection bias \citep{heckman1979sample}.\footnote{For more information see \citet{greene2002limdep}.}

For the selection equation we use an ordered probit model, defined as follows:

\setlength{\belowdisplayskip}{5pt} \setlength{\belowdisplayshortskip}{1pt}
\setlength{\abovedisplayskip}{-6pt} \setlength{\abovedisplayshortskip}{1pt}

\begin{equation}\label{eq:1}
R_{i}^{*} = \alpha_{1} D_{i} + \overrightarrow{\alpha} \overrightarrow{\omega} + \epsilon_{i}
\end{equation}

\setlength{\belowdisplayskip}{11pt} \setlength{\belowdisplayshortskip}{1pt}
\setlength{\abovedisplayskip}{-4pt} \setlength{\abovedisplayshortskip}{1pt}

\begin{equation}\label{eq:2}
R_{i} = 
\begin{cases}
  1 \ \ if \ \ - \infty < R_{i}^{*} \leq \mu_{1} \\
  \vdots \\
  j \ \ if \ \ \mu_{j} < R_{i}^{*} < \infty
\end{cases}
\end{equation}

Where $D_{i}$ is the demand for the student's selected class, $\overrightarrow{\omega}$ is a vector of control variables\footnote{Specifically the variables include Minority, Female, InState, Intl, Needy, AcadRating, URM (Underrepresented Minority), and Subject. See \tablelabel{tab:def1} and \tablelabel{tab:def2} for definitions.}, and $\epsilon_{i}$ is the error term. 
In \eqref{eq:2} we see that the unobserved selection variable $R_{i}^{*}$ corresponds to the observed $R_{i}$ through $\mu$, a vector of unknown cutoffs. 
The variable $j$ represents the number of selection categories there are, where in any of our regressions $j$ is at least two and at most four. 
For example, if $j$ is equal to two, then $R_{i}$ equal to one represents those students who selected into their first choice class and $R_{i}$ equal to two represents those students who did not select into their first choice class (instead selecting into their second choice, third choice, etc.). 
The ordered probit model ultimately estimates the probability that $R_{i}$ is equal to $j$ using $R_{i}^{*}$, that is, 

\setlength{\belowdisplayskip}{5pt} \setlength{\belowdisplayshortskip}{1pt}
\setlength{\abovedisplayskip}{-4pt} \setlength{\abovedisplayshortskip}{1pt}

\begin{equation}\label{eq:3}
Pr(R_{i} = j) \ = \ Pr(\mu_{j-1} < R_{i}^{*} \leq \mu_{j})
\end{equation}

\noindent Once we run the selection equation, we use the inverse mills ratios in the outcome equation.

We use an OLS model very similar to our foundational model as our outcome equation. 
The only difference is that the inverse mills ratios are added as a control variable to correct for selection bias. 
The model is as follows:

\begin{equation}\label{eq:4}
G_{i} = \beta_{0} + \beta_{1} Ability_{i} + \beta_{2} Ability_{i}^{CM1} + \beta_{3} Ability_{i}^{CM2} + \beta_{4} \lambda_{i} + \overrightarrow{\beta} \overrightarrow{z} + \epsilon_{i}
\end{equation}

Where the variables are defined in the same manner as \eqref{eq:0}, and $\lambda_{i}$ is the inverse mills ratios taken from the first stage regression.
Just as in \eqref{eq:0} the OLS regression is run on the subset of students that are not part of the peer measures. 
However, the calculation of different inverse mills ratios for each of the possible selection categories makes it necessary for a separate regression to be run on each of the students who selected into a particular selection category.\footnote{This is why there are Grade1, Grade2, etc. categories in the regressions seen in \sectlabel{results}.}
As an example, suppose $Ability_{i}^{CM1}$ and $Ability_{i}^{CM2}$ are defined as the proportion of high achievers in a class and the proportion of low achievers in a class respectively and the number of selection categories ($j$) is equal to two. 
The outcome equation \eqref{eq:4} will first be regressed on the middle achievers who selected into their first choice class, then on the middle achievers who did not select into their first choice class, producing two sets of regression outputs. 

\subsection{Controlling Selection Bias}\label{methods:csb}

We use two methods in an attempt to control for selection bias, a serious problem in many peer effect models \citep{carman2012classroom,burke2013classroom,ding2007peers}. 
The first method, as mentioned in \sectlabel{data}, is to limit our data to first year students in their first course at the institution.
The incoming classes at the intitution come from a diverse background (both geographically and socioeconomically)\footnote{Compared to other institutions of this type.} so it is highly unlikely that students are selecting their first course based on any prior relationship to classmates.
This effectively eliminates one possible area of selection bias, self-selecting into courses based on prior bonds/relationships.

The second method we use to correct for selection bias is the two stage selection model.
Specifically, our model is controlling for any selection bias that results from students selecting into classes high on their preference list or students selecting into classees low on their preference list.\footnote{The cuttoffs that define a ``high'' and ``low'' preference course are determined by the number of categories ($j$) in our selection model.} 
The idea is that there are unobservable factors that correlate with both student class preferences and the grade received in the class, which would then bias our coefficients of interest ($\beta_{2}$ and $\beta_{3}$). 
For instance, it may be that popular courses draw a higher proportion of high achieving students, while the majority of middle achieving students are drawn into those same courses because they enjoy the subject. 
In this case a higher proportion of high achievers is not causal to the increased grades of middle achievers. 
Instead, an unobservable variable (passion for the subject) drives middle achievers to self select into courses where they will achieve higher grades and there happen to be a higher proportion of high achievers. 
In such a case, the self-selection of middle achieving students would bias our coefficients of interest and our results would be inaccurate.
By using the two stage regression model, we are attempting to correct for this type of selection bias.
