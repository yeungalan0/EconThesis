\documentclass[12pt,a4paper,english,fleqn]{article}
\usepackage{mathptmx}
\usepackage[scaled=0.86]{helvet}
\renewcommand{\familydefault}{\rmdefault}
\usepackage[T1]{fontenc}
\usepackage[latin9]{inputenc}
\usepackage{fancyhdr}
\pagestyle{fancy}
\setcounter{secnumdepth}{2}
\usepackage{setspace}
\usepackage[authoryear,round]{natbib}
\bibliographystyle{agsm}
\usepackage{babel}
\usepackage{rotating}
\usepackage{nameref}
\usepackage{amsmath}
\usepackage[flushleft]{threeparttable}
\usepackage{longtable}
\usepackage{setspace}
\usepackage{breakcites}
\usepackage{afterpage}
\usepackage{chngcntr}
\counterwithin{table}{section}
\usepackage{float}
%\usepackage[outer=25mm,inner=35mm,vmargin=20mm,includehead,includefoot,headheight=15pt,showframe]{geometry}
\usepackage[outer=25mm,inner=35mm,vmargin=20mm,includehead,includefoot,headheight=15pt,headsep=30pt]{geometry}

\renewcommand\headrulewidth{0pt}
%\renewcommand\headheight{17pt}
\newcommand{\prbf}[1]{\textbf{#1}}

% Section variables used throughout
\newcommand{\sectlabel}[1]{section~\ref{#1} (\nameref{#1})}
\newcommand{\tablelabel}[1]{Table~\ref{#1}}

% Caption formatting
\usepackage[font=footnotesize,labelfont=bf]{caption}


% Reduce font size for section
\makeatletter
\renewcommand\section{\@startsection{section}{1}{\z@}
{-5ex \@plus -1ex \@minus -.2ex}{2.3ex \@plus.2ex}{\normalfont\large\bf}}

%Reduce font size for subsection
\renewcommand\subsection{\@startsection{subsection}{2}
{\z@}{-3.25ex\@plus -1ex \@minus -.2ex}{1.5ex \@plus .2ex}{\normalfont\bf}}


% Enforce proper line breaks and avoid widows and orphans
% \tolerance 1414
% \hbadness 1414
% \emergencystretch 1.5em
% \hfuzz 0.3pt
% \widowpenalty = 10000
% \clubpenalty=10000
% \vfuzz \hfuzz
% \raggedbottom

%Set date to empty
\date{02/24/16}

% Format footnotes

% Footnote formatting
\usepackage[hang]{footmisc}
\renewcommand{\hangfootparindent}{1em}
\renewcommand{\hangfootparskip}{0em}
\renewcommand{\footnotemargin}{0.00001pt}
\def\footnotelayout{\hspace{1em}}%

%%%%%%%%%%%%%%%%%%%%%%%%%%%%%%%%%%

% Place your own additions to the preable HERE, that is, BEFORE the following
% call to hyperref

%%%%%%%%%%%%%%%%%%%%%%%%%%%%%%%%%%%

%Hyperlink settings for linkcolors and initial view
\usepackage[%
colorlinks=true,%
linkcolor=black,%
citecolor=black,%
urlcolor=EconomicsDarkBlue,%
pdfstartview=FitH,%
pdfview=FitH,%
pdfpagemode=UseNone]{hyperref}

%Define font for hyperlinks
\def\UrlFont{\normalfont}

% Some fine-tuning of layout
\usepackage{microtype}

%%%%%%%%%%%%%%%%%%%%%%%%%%%%%%%%%%%

% Put your additions to the preable above the call to hyperref, not here, unless you
% want to modify hyperref settings, as is the case with the precvious two commands

% \@ifundefined{showcaptionsetup}{}{%
%  \PassOptionsToPackage{caption=false}{subfig}}
% \usepackage{subfig}
% \AtBeginDocument{
%   \def\labelitemi{\tiny\(\bullet\)}
% }
% \makeatother

\begin{document}

\title{Here is the Title of Your Paper}


\author{Alan Yeung%
\thanks{Give your affiliation and thanks here.%
}.}
\maketitle
\begin{abstract}
\noindent Here comes your abstract.

This is a template for documents prepared with \LaTeX{} for the journal
Economics-The Open Access, Open Assessment E-Journal (www.economics-ejournal.org).

You find further instructions in the Introduction. Please read them.

\noindent \medskip{}


\noindent \emph{Keywords: }Insert the keywords here, separated by
commas.

\noindent \emph{Journal of Economic Literature Classification: }Insert
the classification numbers, such as \ J7, B54,\emph{ }etc. here\emph{.}
\end{abstract}


\newpage{}

\doublespacing

\section{Introduction}\label{intro}

In higher education, economists and prospective students often use the performance measures of the student body as a proxy measure for the quality of the institution \citep{smith2015new,sarmiento2015quality,black2006estimating}.
For example, the average high school GPA, average SAT score, and average class rank are all common reported characteristics of an institution and indicators of institutional quality.
The underlying idea of using the quality of the student body as a measure for the quality of the institution, is that the quality of one's peers matter. 
Indeed, several studies have shown that peer quality has a positive impact on long term student outcomes, such as graduation rates and cumulative GPA \citep{smith2015new,luppino2015college,ost2010role}. 
Naturally, these findings lead to another area of research, identifying the specific areas of a student's life where peer effects are most significant. 
One of the most obvious areas for a strong effect to exist is in the classroom.
Thus, we seek to understand peer effects in the classroom, specifically how the ability of one's classmates impacts a student's class grade.

This study contributes to the literature in two ways.
First, this is the first study of it's kind (known to us) to research peer effects at the classroom level at this type of institution (a small private liberal arts college).
Whereas the majority of other studies focus on K-12 education, we analyze the peer effects at an institution of higher education.
Second, due to our access to data, we are able to use a unique method (at least in terms of classroom peer effects literature\footnote{Most of the literautre uses quasi-random assignments or fixed effects models. See \sectlabel{litreview} for more details.}) to analyze the data and combat selection bias, a serious problem in studies of this type \citep{carman2012classroom,burke2013classroom,ding2007peers}. 

Using data from a small private liberal arts college in the mid-west we find that, regardless of the changes we applied to the peer measure, the proportion of high achieving students in a class has a small but significant negative impact on the grades of middle achieving students. 
Additionally, we find that, when the peer measure is at its strictest level, the proportion of low achieving students in a class has a small but positive impact on the grades of middle achieving students. 
These results are contrary to those found in most of the current literature \citep{kang2007classroom,carman2012classroom,burke2013classroom,schlosser2008inside,lavy2012good}, which may be due to changing peer dynamics from K-12 education to higher education, grade curving occurring in classes at the institution, or a combination of both.
Though these results are interesting, it is unclear how well they generalize to other institutions, of different types and the same type, due to the current scarcity of literature. 

The structure of the paper is as follows: 
We discuss the related literature in section~\ref{litreview}. 
In section~\ref{data}, we examine the data used in this study. 
In section~\ref{methods}, we describe our empirical methodology.
We present our results, discuss the implications, and offer suggestions for further research in section~\ref{results}.
Lastly, we conclude in section~\ref{conclusion}.

\section{Literature Review}\label{litreview}

\citet{smith2015new}, \citet{luppino2015college}, and \citet{ost2010role} show that peer effects have important long term impacts on cumulative GPA and graduation rates at institutions of higher education.
These are important broad findings, which have naturally led to a more narrow focus in the literature.
Much of the current literature focuses on identifying the areas of a student's life where peer effects are most significant. 

For instance, roommate peer effects research has been a trending topic, due to the natural quasi-random roommate assignments made by many institutions of higher education.
Quasi-random roommate assignments, such as those used in \citet{griffith2014peer} and \citet{zimmerman2003peer}, allow researchers to avoid the perils of selection bias and lead to theoretically more accurate results.
However, most of the literature is divided on whether or not roommate peer effects exist \citep{griffith2014peer,zimmerman2003peer,sacerdote2000peer,foster2006s,mcewan2006roommate}.
This is a testament to the difficulty of finding peer effects and the difficulty of solving the econometric problems that exist in peer effects modeling.

Our research adds to the growing (but currently scarce) literature focused on peer effects in the classroom.
In a similar vain to the roommate peer effects literature, much of the literature on peer effects in the classroom exploits quasi-random classroom assignments.
For instance, \citet{kang2007classroom} exploit the quasi-random classroom assignments in South Korean middle schools, and \citet{carman2012classroom} utilize the fact that students are assigned to Chinese middle schools either randomly or through an admissions test.
Both studies focused on determining the significance of peer effects on classroom grades, and find evidence of peer effects.
\citeauthor{kang2007classroom} finds that strong students\footnote{For our purposes, any qualitative measure of students abilities (high or low achiever, strong or weak student, etc.) refers to students relative acadmeic performance (usually on standardized tests). This is standard practice in all of the related literature \citep{carman2012classroom,burke2013classroom,kang2007classroom,schlosser2008inside,lavy2012good}} have a positive impact on the academic performance of other students, while weak students have a negative effect on the performance of other students. 
\citeauthor{carman2012classroom} find somewhat mixed results, but in general they find evidence of positive and significant peer effects on the impact of classmate quality on the academic performance of other students.
Other studies use econometric techniques, such as fixed effects modeling, to correct for any selection bias.
For example, \citet{schlosser2008inside} and \citet{lavy2012good} control for fixed effects when studying the peer effects of underachievers and overachievers on other students in secondary schools.
Both studies find evidence of peer effects where a larger proportion of lower achieving students in a high school class has a negative impact on the achievement of ``regular'' students in the class.
\citeauthor{lavy2012good} in particular find a positive peer effect from high achieving peers on girls.
Additionally, \citet{burke2013classroom} use fixed effects modeling to analyze the peer effects in Florida public school classrooms.
In general, \citeauthor{burke2013classroom} find results in line with the majority of other studies, an increase in the fraction of relatively higher achieving peers makes other students better off and an increase in the fraction of relatively lower achieving peers makes other students worse off.
However, there are a few notable exceptions.
\citeauthor{burke2013classroom} find that high achieving students benefit from an increased fraction of low or high achieving students, and in middle school mathematics, middle achievers benefit from a higher fraction of low achievers.
The authors suggest that the intellectual distance between high achievers and low achievers may be the reason for their former finding.
They explain that the intellectual distance may be so vast as to be a barrier to communication between the two groups, whereas high achievers can communicate effectively with the less intellectually distanced middle achievers (making high achievers worse off). 
Unfortunately, \citeauthor{burke2013classroom} do not provide an explanation for their later finding which resembles one of the results of this paper.
With the few exceptions found by \citet{burke2013classroom}, the majority of the literature finds that a larger proportion of higher achievers in a class has a positive impact on the academic performance of other students, and a larger proportion of low achievers in a class has a negative impact on the academic performance of other students.%\footnote{See the \nameref{results} section for the reasoning.}%

Whereas the majority of other studies focus on K-12 education, we add to the literature by analyzing the peer effects in higher education. 
It is possible that the peer dynamics change in the transition from secondary education to higher education, and our data set gives us a unique opportunity to analyze these peer dynamics.
% Additionally, our data set comes from a small liberal arts institution which should make it an excellent environment to see magnified peer effects.\footnote{For more details see \sectlabel{summarystats}.}

\section{Data}\label{data}

Our data gives us a unique opportunity to study peer effects in higher education.
First, our data comes from an institution with relatively small class sizes which \citet{diette2015class} and \citet{kokkelenberg2008effects} show have a positive impact on average student achievement (measured by grades and test scores).
The primary reasoning for the inverse relationship between achievement and class size is that students have more quality time to interact with teachers and peers as class size decreases, suggesting that we may expect to find magnified peer effects at this institution. 
Additionally, the data on student preferences (Ranking) and class demands (Demand) enable us to use a two stage selection model, a model generally not used in other classroom peer effects literature\footnote{See \sectlabel{litreview} for more details.}, to correct for selection bias and analyze the data. 

The data set comes from a selective medium sized liberal arts college in the mid-west, henceforth referred to as the institution, for which data is collected on five cohorts (2011-2015). 
Classroom level data is used in our regressions as it has been shown to generate stronger results than other group level measures \citep{burke2013classroom}. 

In total there are 1,412 observations and 18 variables. 
Additionally, two different types of data are used in this study. 
The first type is classroom data, that consists of individual student level characteristics as well as classroom characteristics. 
We use data from the first class taken by first year students in order to mitigate selection bias\footnote{Discussed in the \nameref{methods} section.}. 
The second type of data we use is point data, which consists of the number of points students bid on classes and their preferences for certain classes. 

\subsection{Classroom Data}\label{data:classdata}

%We choose to use only data on the first class taken by first year students in order to mitigate selection bias. First year students must select their first course before the semester begins using a point bidding system. Students are accepted into the institution from a variety of areas and backgrounds, and it is very unlikely that students were able to select courses based on any previous relationships. This reduces the likelihood of selection bias effecting our model. In addition, first year students choose courses for a variety of reasons, and these reasons add an element of randomness to the course selection, which also helps to mitigate selection bias. 

The classroom level data on first year students in their first class at the institution is primarily used in the outcome equation.\footnote{Some classroom level data is also used in the selection equation. See \sectlabel{data:pointsdata} for details.}
There are five key variables in our outcome equation, Grade, AcadRating (Academic Rating), PctTopX, PctBotX, and PctMidY. 
The dependent variable, Grade is our performance measure, and is the grade received by a student after taking the course. 
Courses are graded on a four point scale, and there are eleven possible grades ranging from an ``A'' (4.0) to ``F'' (0.0).\footnote{The possible grades are A = 4.0, A- = 3.7, B+ = 3.3, B = 3.0, B- = 2.7, C+ = 2.3, C = 2.0, C- = 1.7, D+ = 1.3, D = 1.0, F = 0.0} 
As a measure of student ability we follow the literature and use a proxy measure developed on data prior to college enrollment, Academic Rating \citep{griffith2014peer,smith2015new}. 
The Academic Rating is a number assigned to all students at the institution. 
It is a number that represents the culmination of a student's high school GPA, their test scores, the difficulty of the high school curriculum, the quality of their high school, and their writing ability. 
As suggested by \citet{betts2003determinants} and \citet{dooley2012persistence} these variables are all common (and ``good'') indicators of college academic performance.
The Academic Rating variable ranges from a low of one (representing a low ability student) to a high of sixty-five (representing a high ability student).

The Academic Rating variable is used to create our peer measure variables PctTopX, PctBotX, and PctMidY. 
PctTopX represents the proportion of students in the class that are in the top X percent of the sample based on Academic Rating. 
For instance, PctTop5 represents the proportion of students in the class that are in the top five percent of the sample based on Academic Rating.
PctBotX is defined similarly for the bottom X percentage in terms of Academic Rating. 
PctMidY is the proportion of students in a class that are not in the top X percent or bottom X percent of the sample in terms of Academic Rating. 
The remaining variables, Minority, Female, InState, Intl (International), Needy, ClassSize, URM (Underrepresented Minority), Year, Division, and Professor, are used as control variables in the regression.
\tablelabel{tab:def1} displays the definitions of all the class level variables used. 

\bigskip

\begin{table}[htb]
\centering
\caption{Classroom Variable Definitions}\label{tab:def1}
 \begin{tabular}{p{0.15\linewidth}|p{0.8\linewidth}} 
 \hline\hline
 Variable & Definition \\ [0.5ex] 
 \hline\hline
Grade & The grade received by a student after taking the course. \\ 
 \hline
 AcadRating & Referred to as Academic Rating, represents the culmination of a student's high school GPA, their test scores, the difficulty of the high school curriculum, the quality of their high school, and their writing ability. \\
 \hline
 PctTopX & The proportion of students in the class in the top X percent of the sample based on Academic Rating \\
 \hline
 PctBotX & The proportion of students in the class in the bottom X percent of the sample based on Academic Rating \\
 \hline
 PctMidY & The proportion of students in a class that are not in the top X percent or bottom X percent of the sample in terms of Academic Rating. \\
 \hline
 Minority & A dummy variable representing whether or not the student is non-Caucasian. 1 = non-Caucasian \& 0 = Caucasian\\
 \hline
 Female & A dummy variable representing whether or not the student identifies as a female. 1 = female \& 0 = male\\
 \hline
 InState & A dummy variable indicating whether or not the student is an in-state student.  1 = in-state \& 0 = out of state\\
 \hline
 Intl & A dummy variable indicating whether or not the student is an international student.  1 = international \& 0 = not international\\
 \hline
 Needy & Whether or not a student qualified for need based financial aid.  1 = financial aid \& 0 = no financial aid\\
 \hline
 ClassSize & An integer the represents the total number of students in a class. \\
 \hline
 URM & Under Represented Minority, a dummy variable indicating whether or not the student is Caucasian/Asian, or not.  1 = Asian or Caucasian \& 0 = other ethnicity \\
 \hline
 Year & The year the class took place. \\
 \hline
 Division & The subject area of the class (Natural Science, Social Sciences, or Humanities). \\
 \hline
 Professor & An identifier for the professor teaching the course. \\
[1ex] 
 \hline\hline
\end{tabular}
\end{table}

\clearpage{}

\subsection{Points Data}\label{data:pointsdata}

At this institution a bidding system is used to ration classes to students. 
From 2011-2014 the bidding system was as follows: 
students are allotted twenty points and must rank eight classes in terms of their preferences, 
after classes are ranked students then must bid a number between zero and twenty (inclusive) points per class on their list, 
students with the highest number of points bid per class are allotted seats, and ties are broken randomly. 
If a student does not make it into any class on his or her preference list, then a class is chosen for the student at random. 

In the year 2015, the bidding system was changed in an effort to allow more students to select into a class higher on their preference list. 
The system was changed in the following ways:
students are allotted 100 points and must rank eight classes in terms of their preferences,
after classes are ranked students then must bid a number between one and twenty (inclusive) points per class on their list.
The remaining rules from the original system are the same. 
This new system effectively forces students to spread their points into multiple classes (whereas in the original system all points could be placed into one class).

The changes to the bidding system affect one of our key variables, Demand. 
The Demand variable is an exclusion restriction in our selection equation that represents the total number of points bid on a course divided by the number of bidders. 
In an attempt to correct for the changes in the bidding system, the 2015 Demand calculations are divided by five, because students receive five times the number of points compared to the original system. 
This corrects the mean of Demand in 2015, however an effect on standard deviation persists. 

Another key variable in our selection equation is Ranking. 
Ranking is the dependent variable in the selection equation, and is a number between one and eight that specifies the student's preference for the course, where one is a high preference, eight is a low preference, and preferences are not repeated. 
The remaining variables in the first stage regression, Minority, Female, InState, Intl (International), Needy, AcadRating, URM (Underrepresented Minority), and Subject are used as control variables. Refer to \tablelabel{tab:def2} for the definitions of the unique variables used in the first stage regression. 

\bigskip

\begin{table}[htb]
  \centering
  \caption{Points Data Variable Definitions}\label{tab:def2}
  \begin{tabular}{p{0.15\linewidth}|p{0.8\linewidth}} 
    \hline\hline
    Variable & Definition \\ [0.5ex] 
    \hline\hline
    Ranking & A number between one and eight specifying the student's preference for the course, where one is a high preference and eight is a low preference. \\ 
    \hline
    Demand & The total number of points bid on a course divided by the number of bidders. For the year 2015, this variable was divided by five to correct for the bidding system changes.\\
    \hline
    Subject & The specific subject of the course, such as mathematics, anthropology, chemistry, psychology, etc. For a full list of subjects see \appendixlabel{appendix:a}. \\
    [1ex] 
    \hline\hline
  \end{tabular}
\end{table}

\tablelabel{tab:freq_Ranking} shows the number of students who selected into their first choice course, second choice course, third choice course, etc. 
It is important to note that the majority, 64\%, of students select into their first choice course, while only 36\% of students selected into a course that is not their first choice.

\begin{table}[H]
\centering
\caption{Student Course Selections (Ranking)}\label{tab:freq_Ranking}
\begin{tabular} {l|r|r}
\hline
\hline
Item           & Number & Per cent \\
\hline
First Choice   & 903    & 64       \\
Second Choice  & 201    & 14       \\
Third Choice   & 136    & 10       \\
Fourth Choice  & 62     & 4        \\
Fifth Choice   & 38     & 3        \\
Sixth Choice   & 36     & 3        \\
Seventh Choice & 16     & 1        \\
Eighth Choice  & 20     & 1        \\
Total          & 1,412  & 100      \\
\hline
\hline
%\multicolumn{3}{@{}l}{\footnotesize{\emph{Source:} }}
\end{tabular}
\end{table}


\subsection{Summary Statistics}\label{summarystats}

The descriptive statistics for the non-dummy variables are displayed in \tablelabel{tab:summarystats}.\footnote{Control variables, such as Year, Professor, Division, and Subject where not summarized.} 
The outcome of interest, Grade, has a mean that changes slightly over time, while the standard deviation remains fairly constant. 
From 2011-2012 the mean Grade was 3.115, then from 2013-2015 the mean grade increased to 3.15. 
This suggests that there might have been some grade inflation over the years as the mean Academic Rating, a measure of student ability, remained fairly consistent from 2011-2014 (jumping by about two points in 2015). 
The variables PctTopX, PctBotX, and PctMidY all vary slightly from their expected values, indicating that the distribution of abilities is not uniform every year. 
That is, one would expect the mean of PctTop5 to always be about 0.05, however this is not the case since the Academic Rating cutoffs are not exact\footnote{Exactly 5\% of students do not have an Academic Rating higher than our cutoff. Instead the number is about 0.048\% and this is true for all of our defined cutoffs.} and the distribution of abilities is not uniform. 
In fact, the distribution of abilities seems to be biased towards recent years, meaning the majority of high achieving students were enrolled in more recent years, as the higher mean Academic Rating, PctTop5, and PctTop10 in 2015 indicate. 
ClassSize jumps from a mean of 11 in 2011-2014 to 14.27 in 2015 because fewer classes were offered and more students were in the incoming class.
The average number of classes offered fell from 29 in 2012-2014 to 25 in 2015.\footnote{In 2011 the mean class size was also 25, however there were fewer students in the incoming class.}
As expected, the Demand variable has a lower standard deviation in 2015 compared to 2011-2014 because of the bidding system changes. 

% It may be important to note that several studies suggest that smaller class sizes have a positive impact on average student achievement (measured by grades and test scores) \citep{diette2015class,kokkelenberg2008effects}. 
% The primary reasoning for the inverse relationship between achievement and class size is that students have more quality time to interact with teachers and peers as class size decreases. 
% As this institution has relatively small class sizes, we may expect to find magnified peer effects. 

\newpage{}

\thispagestyle{plain}
\begin{sidewaystable}[b]
\caption{Summary Statistics}\label{tab:summarystats}
\centering\begin{tabular}{l|c|c|c|c|c|c|c|c|c|c|c|c}
\hline
\hline
& \multicolumn{2}{|c|}{2011 Data} & \multicolumn{2}{|c|}{2012 Data} & \multicolumn{2}{|c|}{2013 Data} & \multicolumn{2}{|c|}{2014 Data}  & \multicolumn{2}{|c|}{2015 Data} & \multicolumn{2}{|c}{Overall} \\
\hline
\prbf{Variable} & \prbf{Mean} & \prbf{Std. Dev.} & \prbf{Mean} & \prbf{Std. Dev.} & \prbf{Mean} & \prbf{Std. Dev.}  & \prbf{Mean} & \prbf{Std. Dev.} & \prbf{Mean} & \prbf{Std. Dev.} & \prbf{Mean} & \prbf{Std. Dev.} \\
\hline
            Grade                    & 3.12  & 0.96 & 3.11  & 0.87 & 3.17  & 0.78 & 3.15  & 0.81 & 3.16  & 0.87 & 3.14  & 0.86 \\
            AcadRa{\textasciitilde}g & 50.72 & 6.09 & 50.89 & 5.9  & 49.98 & 6.25 & 50.95 & 6.44 & 52.80 & 6.19 & 51.12 & 6.26 \\
            PctTopQ                  & 0.30  & 0.13 & 0.28  & 0.17 & 0.22  & 0.17 & 0.27  & 0.15 & 0.41  & 0.14 & 0.30  & 0.17 \\
            PctTop5                  & 0.08  & 0.10 & 0.06  & 0.10 & 0.06  & 0.07 & 0.04  & 0.07 & 0.14  & 0.10 & 0.08  & 0.10 \\
            PctTop10                 & 0.16  & 0.12 & 0.14  & 0.13 & 0.14  & 0.11 & 0.16  & 0.13 & 0.26  & 0.13 & 0.18  & 0.13 \\
            PctBotQ                  & 0.23  & 0.11 & 0.19  & 0.11 & 0.28  & 0.17 & 0.18  & 0.10 & 0.13  & 0.10 & 0.20  & 0.13 \\
            PctBot5                  & 0.04  & 0.06 & 0.05  & 0.08 & 0.06  & 0.12 & 0.06  & 0.08 & 0.04  & 0.06 & 0.05  & 0.08 \\
            PctBot10                 & 0.08  & 0.08 & 0.07  & 0.08 & 0.10  & 0.14 & 0.08  & 0.08 & 0.06  & 0.08 & 0.08  & 0.09 \\
            PctMid50                 & 0.47  & 0.16 & 0.54  & 0.16 & 0.50  & 0.17 & 0.54  & 0.14 & 0.46  & 0.14 & 0.50  & 0.16 \\
            PctMid90                 & 0.87  & 0.11 & 0.89  & 0.12 & 0.87  & 0.14 & 0.90  & 0.12 & 0.82  & 0.10 & 0.87  & 0.12 \\
            PctMid80                 & 0.77  & 0.14 & 0.78  & 0.14 & 0.76  & 0.16 & 0.76  & 0.16 & 0.68  & 0.13 & 0.75  & 0.15 \\
            Minority                 & 0.21  & 0.41 & 0.37  & 0.48 & 0.46  & 0.50 & 0.35  & 0.48 & 0.39  & 0.49 & 0.36  & 0.48 \\
            Female                   & 0.47  & 0.50 & 0.55  & 0.50 & 0.53  & 0.50 & 0.51  & 0.50 & 0.54  & 0.50 & 0.52  & 0.50 \\
            InState                  & 0.15  & 0.35 & 0.13  & 0.33 & 0.15  & 0.36 & 0.12  & 0.33 & 0.15  & 0.36 & 0.14  & 0.35 \\
            Intl                     & 0.02  & 0.15 & 0.01  & 0.08 & 0.06  & 0.23 & 0.05  & 0.22 & 0.07  & 0.25 & 0.04  & 0.20 \\
            Needy                    & 0.39  & 0.49 & 0.36  & 0.48 & 0.36  & 0.48 & 0.32  & 0.47 & 0.43  & 0.50 & 0.37  & 0.48 \\
            ClassS{\textasciitilde}e & 10.85 & 3.09 & 11.91 & 3.05 & 11.11 & 2.48 & 11.39 & 2.62 & 14.27 & 2.47 & 12.00 & 3.01 \\
            URM                      & 0.11  & 0.31 & 0.29  & 0.45 & 0.36  & 0.48 & 0.27  & 0.44 & 0.30  & 0.46 & 0.27  & 0.45 \\
            Ranking                  & 1.77  & 1.54 & 1.70  & 1.44 & 1.79  & 1.41 & 1.91  & 1.79 & 2.14  & 1.41 & 1.87  & 1.53 \\
            Demand                   & 2.81  & 0.83 & 2.62  & 0.75 & 2.28  & 0.86 & 2.37  & 0.86 & 2.44  & 0.22 & 2.49  & 0.76 \\
\hline
            N                        & 226   &      & 292   &      & 290   &      & 285   &      & 319   &      & 1,412        \\
\hline
\hline
\end{tabular}
\end{sidewaystable}

\clearpage{}

\section{Empirical Methodology}\label{methods}

Our original peer effects model is as follows:
\setlength{\belowdisplayskip}{6pt} \setlength{\belowdisplayshortskip}{1pt}
\setlength{\abovedisplayskip}{-4pt} \setlength{\abovedisplayshortskip}{1pt}

\begin{equation}\label{eq:0}
G_{i} = \beta_{0} + \beta_{1} Ability_{i} + \beta_{2} Ability_{i}^{CM1} + \beta_{3} Ability_{i}^{CM2} + \overrightarrow{\beta} \overrightarrow{z} + \epsilon_{i}
\end{equation}

Where $\epsilon_{i} \stackrel{iid}{\sim} N(0,\sigma_{\epsilon}^2)$, $G_{i}$ is the grade received by student $i$ in their first course at the institution, $Ability_{i}$ is a proxy for the student's academic ability (Academic Rating)\footnote{For more information see \tablelabel{tab:def1}.}, $Ability_{i}^{CM1}$ and $Ability_{i}^{CM2}$ are classmate ability measures, and $\overrightarrow{z}$ is a vector of control variables.\footnote{Control variables included Minority, Female, InState, International, Needy, Class Size, URM, Year, Division, and Professor. See \tablelabel{tab:def1} and \tablelabel{tab:def2} for definitions.}
$Ability_{i}^{CM1}$ and $Ability_{i}^{CM2}$ are our peer measures and are defined as one of the following, the proportion of high achieving, middle achieving, or low achieving students in a class\footnote{Unfortunately due to data limitations only the proportion of high achievers and low achievers in a class is used for our peer measures, for more details see \sectlabel{results}.}(as defined by cutoffs in academic rating), and $Ability_{i}^{CM1}$ is not the same measure as  $Ability_{i}^{CM2}$. 
Thus $\beta_{2}$ and $\beta_{3}$ are the primary coefficients of interest, as they estimate the impact of the classmates ability variables (our peer measures) on a student's grade. 
The model \eqref{eq:0} is regressed on the subset of students that are not incorporated in the peer measures. 
For example, if $Ability_{i}^{CM1}$ and $Ability_{i}^{CM2}$ are defined as the proportion of high achievers in a class and the proportion of low achievers in a class respectively, then the model would be regressed on the middle achievers. 

However, since our sample is nonrandom\footnote{Unfortunately students are not assigned to classes and institutions randomly, instead the institution selects specific students, students next select the institution, and students then select into classes.} the calculated coefficients of model \eqref{eq:0} are at risk of being bias \citep{heckman1979sample}.  
This bias, known as selection bias, is a serious threat to peer effects models of this type because students (or in some cases parents) often get a choice in selecting an institution to attend \citep{carman2012classroom,burke2013classroom,ding2007peers}. 
This selection, as discussed by \citet{kang2007classroom}, often leads to a non-random sample because students of similar personal and family backgrounds often select into the same institutions.
We are concerned about a similar self-selection, but at the classroom level, that is, students of similar personal and family backgrounds may select into similar classes, creating a non-random sample in each classroom and biasing our results.
Therefore, we must update our model to correct for any selection bias that may exist.

\subsection{Two Stage Selection Model}\label{methods:tssmodel}

We control for two types of selection bias. 
The first type of selection bias is due to selection into courses based on previous relationships. 
Some students may be selecting into courses because they have intelligent friends in those courses who may be willing to help them.
In this case, the coefficients would be bias because an unobservable factor (previous relationships) is correlated with both the course grade and classmate ability levels (our peer measure). 
We control for this by limiting our data to first year students in their first course at the institution.
The incoming classes at the institution come from a diverse background, both geographically and socioeconomically, so it is highly unlikely that students are selecting their first course based on any prior relationship to classmates.
This effectively eliminates the first type of selection bias, self-selecting into courses based on prior bonds/relationships.

The second type of selection bias we control for is students selecting into courses where they are likely to perform well (or poorly), due to unobservable factors. 
For instance, some students may intentionally select into courses where they are likely to receive a high grade, due to unobservable factors (interest in the course, previous experience in the subject, etc.), that happen to also attract students of a certain ability level.
In this case, the coefficients would be bias because unobservable factors are correlated with both the course grade and classmate ability levels (our peer measure).
We control for this type of selection bias by using a two stage selection model.
Specifically, our model is controlling for any selection bias that results from students selecting into courses that are high or low on their preference list.\footnote{The cutoffs that define a ``high'' and ``low'' preference course are determined by the number of categories ($j$) in our selection model.} 
The idea is that students that have unobservable characteristics that would lead them to do well in certain courses will have higher preferences for those courses, whereas students who have unobservable characteristics that would lead them to do less well in a course (lack of interest, disappointment about not getting into a high preference course, etc.) will have low preferences for the course.
By controlling for student course preferences we are controlling for some of the unobservable factors that, as a result of student self-selection into courses, may bias our coefficients.

As discussed in \sectlabel{litreview}, most studies of this kind exploit quasi-random student class assignments or utilize fixed effects models to control for selection bias \citep{kang2007classroom,carman2012classroom,schlosser2008inside,lavy2012good}. 
However, due to our unique access to student preference data, we use a two stage selection model to correct for selection bias, similar to the one described in \citet{heckman1979sample}. 
Our two stage selection model uses an ordered probit model in the first stage (the selection equation) and OLS in the second stage (the outcome equation). 
From the first stage ordered probit model we take the calculated inverse mills ratios and use them as a control variable in the second stage outcome equation. 
The inverse mills ratios are calculated estimates, that when used in the outcome equation, help to control for selection bias \citep{heckman1979sample}.\footnote{For more information see \citet{greene2002limdep}.}

For the selection equation we use an ordered probit model, defined as follows:

\setlength{\belowdisplayskip}{5pt} \setlength{\belowdisplayshortskip}{1pt}
\setlength{\abovedisplayskip}{-6pt} \setlength{\abovedisplayshortskip}{1pt}

\begin{equation}\label{eq:1}
R_{i}^{*} = \alpha_{1} D_{i} + \overrightarrow{\alpha} \overrightarrow{\omega} + \epsilon_{i}
\end{equation}

\setlength{\belowdisplayskip}{11pt} \setlength{\belowdisplayshortskip}{1pt}
\setlength{\abovedisplayskip}{-4pt} \setlength{\abovedisplayshortskip}{1pt}

\begin{equation}\label{eq:2}
R_{i} = 
\begin{cases}
  1 \ \ if \ \ - \infty < R_{i}^{*} \leq \mu_{1} \\
  \vdots \\
  j \ \ if \ \ \mu_{j} < R_{i}^{*} < \infty
\end{cases}
\end{equation}

Where $D_{i}$ is the demand for the student's selected course, $\overrightarrow{\omega}$ is a vector of control variables\footnote{Specifically the variables include Minority, Female, InState, Intl, Needy, AcadRating, URM (Underrepresented Minority), and Subject. See \tablelabel{tab:def1} and \tablelabel{tab:def2} for definitions.}, and $\epsilon_{i}$ is the error term. 
In \eqref{eq:2} we see that the unobserved selection variable $R_{i}^{*}$ corresponds to the observed $R_{i}$ through $\mu$, a vector of unknown cutoffs. 
The variable $j$ represents the number of selection categories, where in any of our regressions $j$ is at least two and at most four. 
For example, if $j$ is equal to two, then $R_{i}$ equal to one represents those students who selected into their first choice course and $R_{i}$ equal to two represents those students who did not select into their first choice course (instead selecting into their second choice, third choice, etc.). 
The ordered probit model estimates the probability that $R_{i}$ is equal to $j$ using $R_{i}^{*}$, that is, 

\setlength{\belowdisplayskip}{5pt} \setlength{\belowdisplayshortskip}{1pt}
\setlength{\abovedisplayskip}{-4pt} \setlength{\abovedisplayshortskip}{1pt}

\begin{equation}\label{eq:3}
Pr(R_{i} = j) \ = \ Pr(\mu_{j-1} < R_{i}^{*} \leq \mu_{j})
\end{equation}

\noindent Once we estimate the selection equation, we include the inverse mills ratios in the outcome equation.

We use a model very similar to our original model \eqref{eq:0} as our outcome equation. 
The only difference is that the inverse mills ratios ($\lambda$) are added as a control variable to correct for selection bias. 
The model is as follows:

\begin{equation}\label{eq:4}
G_{i} = \beta_{0} + \beta_{1} Ability_{i} + \beta_{2} Ability_{i}^{CM1} + \beta_{3} Ability_{i}^{CM2} + \beta_{4} \lambda_{i} + \overrightarrow{\beta} \overrightarrow{z} + \epsilon_{i}
\end{equation}

Where the variables are defined in the same manner as \eqref{eq:0}, and $\lambda_{i}$ is the inverse mills ratios taken from the first stage regression.
Just as in \eqref{eq:0} the model \eqref{eq:4} is regressed on the subset of students that are not part of the peer measures. 
However, the calculation of different inverse mills ratios for each of the possible selection categories makes it necessary for a separate regression to be estimated on each of the students who selected into a particular selection category.\footnote{This is why there are Grade1, Grade2, etc. categories in the regressions seen in \sectlabel{results}.}
As an example, suppose $Ability_{i}^{CM1}$ and $Ability_{i}^{CM2}$ are defined as the proportion of high achievers in a course and the proportion of low achievers in a course respectively and the number of selection categories ($j$) is equal to two. 
The outcome equation \eqref{eq:4} will first be regressed on the middle achievers who selected into their first choice course, then on the middle achievers who did not select into their first choice course, producing two sets of regression outputs. 

\section{Results}\label{results}

We begin the section with an analysis of the results from our main regressions, where we find the most prominent results are seen by the impact of high achievers on the grades of middle achievers. 
The results lend significant evidence to the idea that a higher proportion of high achievers in a classroom has a negative impact on the final grade of a middle achieving student. 
We also find evidence indicating that a higher proportion of low achievers in a class has a positive impact on the grades of middle achieving students. 
These results are contradictory to those found in the current literature \citep{kang2007classroom,carman2012classroom,burke2013classroom,schlosser2008inside,lavy2012good} and we discuss some possible explanations for this below. 
Next, we analyze the threat of selection bias to our original OLS model by comparing our overall results. 
We conclude by evaluating the significance of our results. 
Overall, we find that the estimated coefficients are too small to have a significant impact on final student grades, however if these trends continue in the long run then it may be possible that these trends impact cumulative GPA. 

The main results of this paper will focus on the impact of high achievers and low achievers on the grades of middle achievers. 
Unfortunately, data limitations made it impossible to run regressions on the high achievers and low achievers (for any of our peer measures). 
There were simply not enough data points within either group to run the two stage regression model. 
However, the regressions run on the middle achievers still make an important contribution because the majority of students are middle achievers, so we are finding the peer effects that affect most of the population.

\subsection{Regression Analysis}\label{results:reganalysis}

Table 4.1, Table 4.2, and Table 4.3 show the relevant results\footnote{The results of the regression for all the variables can be found in Appendix A. (TO BE ADDED)} of the two stage selection model\footnote{Outlined in \sectlabel{methods}} run on the middle achievers, where the first stage is a two category ordered probit, three category ordered probit, and four category ordered probit respectively. 
Each table contains the relevant results from three regressions where the major difference between each regression within a table is the measure of the ``peer'' used. 
For the regressions found in Table 4.1, 4.2, and 4.3, we use the proportion of high achievers and low achievers in a class as a peer measure and regress on middle achieving students. 
After each regression, the measures used for a high and low achieving peer become stricter (and therefore the measure for the middle achievers becomes less strict). 
For instance, regression 1 (labeled Top/Bottom 25\%) in Table 4.1 focuses on the impact of the proportion of high achievers, defined as the twenty-five percent with the highest academic rating in the sample, and the proportion of low achievers, the twenty-five percent with the lowest academic rating in the sample, on the grades of the middle achievers (those that are not high or low achievers). 
In each regression, GradeX refers to the regression run on the middle achievers that selected into their X choice class based on the first stage ordered probit.\footnote{For more details see \sectlabel{methods:tssmodel}} 

\clearpage{}

\begin{table}[htb]
  \centering
  \begin{threeparttable}
    \caption{Two Selection Categories}\label{tab:2sc}
    \begin{tabular}{l l l l} 
      \hline
      \hline
               & Top/Bott{\textasciitilde}25\% & Top/Bott{\textasciitilde}10\% & Top/Bott{\textasciitilde}5\% \\
               & (Std. Err.)                   & (Std. Err.)                   & (Std. Err.)                  \\
      \hline
      Grade1   &                               &                               &                              \\
      PctBotQ  & -0.150                        &                               &                              \\
               & (0.53)                        &                               &                              \\
      PctTopQ  & -0.787**                      &                               &                              \\
               & (0.41)                        &                               &                              \\
      PctBot10 &                               & 0.612                         &                              \\
               &                               & (0.54)                        &                              \\
      PctTop10 &                               & -0.741***                     &                              \\
               &                               & (0.33)                        &                              \\
      PctBot5  &                               &                               & 1.001**                      \\
               &                               &                               & (0.60)                       \\
      PctTop5  &                               &                               & -0.702**                     \\
               &                               &                               & (0.41)                       \\
      \hline
      Grade2   &                               &                               &                              \\
      PctBotQ  & 1.720**                       &                               &                              \\
               & (1.00)                        &                               &                              \\
      PctTopQ  & 0.660                         &                               &                              \\
               & (0.70)                        &                               &                              \\
      PctBot10 &                               & 0.370                         &                              \\
               &                               & (0.78)                        &                              \\
      PctTop10 &                               & 0.827**                       &                              \\
               &                               & (0.49)                        &                              \\
      PctBot5  &                               &                               & -0.061                       \\
               &                               &                               & (0.70)                       \\
      PctTop5  &                               &                               & 0.198                        \\
               &                               &                               & 0.198                        \\
      \hline
      \hline
    \end{tabular}
    \begin{tablenotes}
    \item{* p<.2, ** p<.1, *** p<.05 \\Note: Regressions run using 3 different peer measures and a two choice ordered probit for the first stage.}
    \end{tablenotes}
  \end{threeparttable}
\end{table}

\clearpage{}

\begin{table}[htb]
  \centering
  \begin{threeparttable}
    \caption{Three Selection Categories}\label{tab:3sc}
    \begin{tabular}{l l l l} 
      \hline
      \hline
               & Top/Bott{\textasciitilde}25\% & Top/Bott{\textasciitilde}10\% & Top/Bott{\textasciitilde}5\% \\
               & (Std. Err.)                   & (Std. Err.)                   & (Std. Err.)                  \\
      \hline
      Grade1   &                               &                               &                              \\
      PctBotQ  & -0.314                        &                               &                              \\
               & (0.52)                        &                               &                              \\
      PctTopQ  & -0.840***                     &                               &                              \\
               & (0.39)                        &                               &                              \\
      PctBot10 &                               & 0.602                         &                              \\
               &                               & (0.54)                        &                              \\
      PctTop10 &                               & -0.705***                     &                              \\
               &                               & (0.33)                        &                              \\
      PctBot5  &                               &                               & 1.008**                      \\
               &                               &                               & (0.60)                       \\
      PctTop5  &                               &                               & -0.667*                      \\
               &                               &                               & (0.41)                       \\
      \hline
      Grade2   &                               &                               &                              \\
      PctBotQ  & 1.248                         &                               &                              \\
               & (1.41)                        &                               &                              \\
      PctTopQ  & 1.614**                       &                               &                              \\
               & (0.92)                        &                               &                              \\
      PctBot10 &                               & -0.138                        &                              \\
               &                               & (1.02)                        &                              \\
      PctTop10 &                               & 0.439                         &                              \\
               &                               & (0.70)                        &                              \\
      PctBot5  &                               &                               & -0.819                       \\
               &                               &                               & (0.88)                       \\
      PctTop5  &                               &                               & -0.585                       \\
               &                               &                               & (1.00)                       \\
      \hline
      Grade3   &                               &                               &                              \\
      PctBotQ  & 1.215                         &                               &                              \\
               & (1.54)                        &                               &                              \\
      PctTopQ  & -0.961                        &                               &                              \\
               & (1.16)                        &                               &                              \\
      PctBot10 &                               & 0.002                         &                              \\
               &                               & (1.18)                        &                              \\
      PctTop10 &                               & 0.541                         &                              \\
               &                               & (0.63)                        &                              \\
      PctBot5  &                               &                               & -0.170                       \\
               &                               &                               & (1.13)                       \\
      PctTop5  &                               &                               & -0.031                       \\
               &                               &                               & (0.83)                       \\
      \hline
      \hline
    \end{tabular}
    \begin{tablenotes}
    \item{* p<.2, ** p<.1, *** p<.05 \\Note: Regressions run using 3 different peer measures and a three choice ordered probit for the first stage.}
    \end{tablenotes}
  \end{threeparttable}
\end{table}

\clearpage{}

\begin{table}[htb]
  \centering
  \begin{threeparttable}
    \caption{Four Selection Categories}\label{tab:4sc}
    \begin{tabular}{l @{\hskip 1in} l} 
      \hline
      \hline
              & Top/Bott{\textasciitilde}5\% \\
              & (Std. Err.)                  \\
      \hline
      Grade1  &                              \\
      PctBot5 & 0.998**                      \\
              & (0.61)                       \\
      PctTop5 & -0.669*                      \\
              & (0.41)                       \\
      \hline
      Grade2  &                              \\
      PctBot5 & -0.791                       \\
              & (0.88)                       \\
      PctTop5 & -0.530                       \\
              & (1.01)                       \\
      \hline
      Grade3  &                              \\
      PctBot5 & 2.443*                       \\
              & (1.64)                       \\
      PctTop5 & -0.215                       \\
              & (1.52)                       \\
      \hline
      Grade4  &                              \\
      PctBot5 & -0.417                       \\
              & (1.72)                       \\
      PctTop5 & -2.847***                    \\
              & (1.18)                       \\
      \hline
      \hline
    \end{tabular}
    \begin{tablenotes}
    \item{* p<.2, ** p<.1, *** p<.05 blah \\Note: Regressions run using 3 different peer measures and a three choice ordered probit for the first stage.}
    \end{tablenotes}
  \end{threeparttable}
\end{table}

\clearpage{}

When looking at the results in Table 4.1, 4.2, and 4.3 there are a few important patterns to note. 
First, the majority of the statistically significant results are found in students who selected into their first choice (Grade1), and students who selected into their second choice or below (Grade2, Grade3, etc.) show inconsistent results. 
Next, all the significant results in Grade1 are consistent, in that the proportion of high achievers has a negative impact on the grades of middle achievers and the proportion of low achievers has a positive impact on the grades of middle achievers. 
Furthermore, the majority of significant results appear as the peer measure becomes stricter, that is more significant results appear as the peer measure changes from the top/bottom twenty five percent, to ten percent, and finally to five percent. 

First, we'll explore the possible reasons behind the inconsistent results in the regressions run on students who selected into a class that was not their first choice. 
One reason, may be that for students who do not select into their first choice class, the peer effect dynamics change and aren't as strong. 
Perhaps the students who did not select into their first choice class have less motivation (or something of the sort) to perform in their non-first choice class, and this leads to less peer interactions (or interactions of a different kind) that ultimately lead to smaller/unmeasurable peer effects. 
Another explanation, one that we believe is more likely, is that there are simply fewer students who select into a course that is not their first choice. 
In the best cases the number of people who do not select into their first choice is about one half of the number of people who do select into their first choice.\footnote{TODO: See Table X and \sectlabel{data} for more details} 
That is, for a two option ordered probit selection model (used in Table 4.1), the number of students who are regressed in the Grade1 calculations (those who selected into their first choice class) is twice that of the number of students are regressed in the Grade2 calculations (those who did not select into their first choice class).\footnote{For more details see \sectlabel{methods}} 
Additionally, as the number of options in the selection model increases (as is the case in Table 4.2 and 4.3) the number of students who did not select into their first choice class is further divided into those who selected into their second choice class, third choice class, etc. and regressed on. 
This essentially means that the regressions run on the students who did not select into their first choice class (Grade2, Grade3, Grade4) have far fewer data points than the regression run on students who did select into their first choice class (Grade1). 
This is a troubling problem as it suggests that the results from the Grade2, Grade3, and Grade4 regressions are far more unreliable than the results of the Grade1 regressions. 
However, the only way to resolve this problem is to collect several more years of data, which is not feasible at this time. 
Therefore, we will focus on the Grade1 regression results for the remainder of the section, and since the majority of students select into their first choice class, they are arguably the most relevant results regardless. 
 
The Grade1 results in Table 4.1, 4.2, and 4.3 show a consistent pattern. 
The proportion of high achievers in a classroom has a negative impact on the grade of middle achievers, regardless of the strictness of the peer measure. 
As the peer measure becomes stricter (specifically at the five percent threshold), we see that the proportion of low achievers in a classroom has a positive impact on the grades of middle achievers. 
It is interesting to note that there is some precedence for using the five percent threshold and finding peer impact results. 
In Lavy et al. (2012) researchers also found strong peer effects using the top and bottom five percent as their peer measures. 
The researchers in Lavy et al. (2012) argue that their use of the top and bottom five percent is not arbitrary by showing that it is precisely those students around the five percent threshold that have a strong peer impact on fellow students. 
That is, they showed that students in the middle ninety percent of the ability distribution do not show strong peer effects of any sort, while the top and bottom five percent are the students who are the most influential. 

Our results run counter to those found in the current literature (granted that the current literature is rather scarce especially in the area of higher education). 
Most of the current literature finds that low achievers hurt the grades of middle achieving students and high achievers help the grades of middle achieving students. 
The reasoning being that low achieving students may disrupt learning (possibly by instilling bad study habits in the middle achieving students) and the high achieving students may facilitate learning (possibly by asking more relevant questions or helping to tutor middle achieving students). 
This kind of reasoning is what many people would expect to happen (and has been discovered in other areas of peer effects literature) \citep{griffith2014peer,zimmerman2003peer,sacerdote2000peer}, which makes our results of the contrary all the more puzzling. 
It is important to note that most of the literature does not focus on higher education, instead choosing to focus on K-12 education in order to reduce any bias from self-selection. 
Therefore, it is reasonable to suppose that peer effects at institutions of higher education may differ from those seen in K-12 education. 
We have developed two possible explanations for these results. 
One possibility is that since most of the previous literature does not focus on courses in higher education, peer dynamics have changed in the transition from K-12 education to undergraduate education. 
It is possible that high achieving students in college have a negative peer effect on grades, perhaps through demoralizing other students, and low achieving students have a positive peer effect on grades, perhaps through increasing the number of group study sessions. 
Another possibility is that grade curving is occurring in many of the classes, and is overpowering any actual peer effects. 
That is to say, a student's class grade is determined by her relative performance to her classmates instead of by her absolute performance. 
Unfortunately, due to the time span of the data and faculty changes, it is not feasible to uncover which classes were truly graded on a curve (in which case these findings would not be surprising) and which classes were not. 
Furthermore, if grade curving is the underlying cause of the results we are finding, then it is still possible that peer effects that are in line with the literature exist, but the curving effects are simply overwhelming any peer effects that are occurring. 

Another interesting trend is that our results are somewhat imbalanced. That is to say, high achieving students clearly have a negative impact on middle achieving students throughout the majority of our regressions, but low achieving students only have a statistically significant positive impact on middle achieving students when the peer measure becomes stricter. 
One explanation may be that it was pure statistical chance that top achievers had a significant negative affect more often than low achievers had a significant positive affect. 
Our sample size of 903 students who selected into their first choice class is not as large as some found in the literature (which reach into the many thousands) \citep{kang2007classroom,lavy2012good}. 
Another possibility is that top students are very effective at having a negative impact on student grades, and this result is simply robust to changes in the peer measure. 
Finally, if grade curving is the underlying cause of our results, it may be that high achieving students dull out a middle achieving student's performance more than a low achieving student helps a middle achieving student shine. 
Since professors may curve grades in a subjective fashion, they may be effected by this psychological phenomenon. 
Currently we do not have an explanation for this phenomena, although there is evidence of similar kinds of psychological imbalance in other areas. 
For example take the well documented theory of loss aversion. 
As explained by Tversky and Kahneman (1991), loss aversion implies that people experience greater impact from a loss when compared to a gain equal in magnitude, and this phenomena has been found in many different areas of human behavior \citep{shalev2002loss,goette2004loss}. 
It is possible that the same psychological mechanisms driving the imbalance in losses and gains (loss aversion) are also driving the imbalance in the effect of the proportion of high achievers and low achievers on a middle achievers grades. 

\subsection{Threat of Selection Bias}\label{results:tsb}

In order for selection bias (a common problem in peer effect models) to affect our model, one must argue that there exists unobservable factors that correlate with both course interests and academic performance. 
This is a challenging argument to make, but nonetheless it must be considered as a serious threat to our model. 
Thus, to correct for any selection bias we use a two stage selection model based on the technique described by Heckman (1979). 
However, even after correcting for selection bias, it seems that our overall results are quite similar to those found when using standard OLS regressions. 
This suggests that selection bias is most likely not affecting our model to any significant degree. 

\subsection{Significance of Results}\label{results:sor}

The above results indicate the presence of an interesting trend at this institution, but should the students or administration even care if this trend exists? In order to shed some light on the implications of our results, we must analyze the magnitudes of the estimated coefficients. 
The average class size in our dataset is twelve students (which has remained fairly constant throughout the time period of our dataset), and if we define high achievers as the top five percent of students then there is one student per class that is a high achiever,  assuming a uniform distribution of high achievers. 
At the five percent threshold, our results indicate that for every ten percent increase in the proportion of high achievers in the classroom the GPA of a middle achiever falls by roughly 0.072 points. 
In order for the proportion of high achievers in an average class to be high enough to have a significant\footnote{Defined as at least changing the sign of a student's grade which takes about 0.3 grade points.} influence on the grades of middle achieving students, forty percent of the average class (five students) must be high achievers.\footnote{0.4 * 0.72 = 0.288} Using the five percent threshold, there are two classes (about 1.5 percent) in our entire dataset which have a proportion of high achievers of at least forty percent. 
It is clear that having a class consist of at least forty percent high achievers is an unlikely scenario, and should not be a concern for most students. 
Similar calculations can be done to show that having a high enough number of low achieving students in a classroom to significantly affect grades is also quite uncommon. 
However, two considerations must be taken into account. 
First, if these results continued in further classes, then the cumulated effect on the GPA may be significant. 
Depending on a student's degree path, they may be more likely to select into classes with a high proportion of low achieving students or classes with a high proportion of high achieving students. 
Higher level classes at the institution may number under 10 students (depending on the popularity of the degree), therefore if these trends continue it is possible for the cumulative GPA (as well as class grades) of a student to be affected. 
This may be a serious problem in unpopular degrees with abnormally small class sizes. 
In addition, the coefficients are estimates, and it may be that the real coefficients are much larger or smaller than those estimated. 
The calculations above show that even though these trends exist, in the short run, the impact on final grades in the first course is most likely not a concern to most students. 
However, more research need to be conducted to determine the long run implications of these results. 



\section{Conclusion}\label{conclusion}

% We found peer effects, but as to how generalizable they are or how large they are we cannot be certain

Several studies have found that the quality of ones peers at an institution of higher learning has a positive impact on long term outcomes, such as graduation rates and cumulative GPA \citep{smith2015new,luppino2015college,ost2010role}.
We seek to answer the question of whether or not peer quality matters in the short term, particularly at the classroom level, and find small but significant effects from the proportion of high achievers and low achievers in a class on the grades of middle achievers. 
The results indicate that the proportion of high achievers in a class has a negative impact on the grades of middle achievers, while the proportion of low achievers in a class has a positive impact on the grades of middle achievers. 
These results are novel and run counter to those found in the majority of the current literature \citep{kang2007classroom,carman2012classroom,burke2013classroom,schlosser2008inside,lavy2012good}, which may be due to changing peer dynamics from K-12 education to higher education, grade curving occurring in classes at the institution, or a combination of both.
 
Due to data limitations we are unable to uncover how classmate ability impacts the grades of high and low achievers, which makes it challenging to prescribe any practical policy recommendations.
Additionally, the type of institution studied was relatively unique, and it is not clear how generalizable these results are. 
Fortunately, these study limitations create promising avenues for future research.
Particularly promising areas of future research include comparing the classroom level peer effects at different types of institutions (such as community colleges or large universities), and, following in the footsteps of \citet{oosterbeek2014gender}, looking at classroom peer effects from a gender peer effects perspective.
       
\newpage{}
\pagestyle{plain}
\singlespacing

\bibliography{references}

\end{document}
