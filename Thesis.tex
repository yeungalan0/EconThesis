\documentclass[12pt,a4paper,english,fleqn]{article}
\usepackage{mathptmx}
\usepackage[scaled=0.86]{helvet}
\renewcommand{\familydefault}{\rmdefault}
\usepackage[T1]{fontenc}
\usepackage[latin9]{inputenc}
\usepackage{fancyhdr}
\pagestyle{fancy}
\setcounter{secnumdepth}{2}
\usepackage{setspace}
\usepackage[authoryear,round]{natbib}
\bibliographystyle{agsm}
\usepackage{babel}
\usepackage{rotating}
\usepackage{nameref}
\usepackage{amsmath}
\usepackage[flushleft]{threeparttable}
\usepackage{longtable}
\usepackage{setspace}
\usepackage{chngcntr}
\counterwithin{table}{section}

\renewcommand\headrulewidth{0pt}
\renewcommand\headheight{17pt}
\newcommand{\prbf}[1]{\textbf{#1}}

% Section variables used throughout
\newcommand{\sectlabel}[1]{section~\ref{#1} (\nameref{#1})}
\newcommand{\tablelabel}[1]{Table~\ref{#1}}

% \newcommand{\introduction}{section 1 (Introduction)}
% \newcommand{\litreview}{section 2 (Literature Review)}
% \newcommand{\data}{section 3 (Data)}
% \newcommand{\introduction}{section 3.1 (Class Room Data)}

% Caption formatting
\usepackage[font=footnotesize,labelfont=bf]{caption}


% Reduce font size for section
\makeatletter
\renewcommand\section{\@startsection{section}{1}{\z@}
{-5ex \@plus -1ex \@minus -.2ex}{2.3ex \@plus.2ex}{\normalfont\large\bf}}

%Reduce font size for subsection
\renewcommand\subsection{\@startsection{subsection}{2}
{\z@}{-3.25ex\@plus -1ex \@minus -.2ex}{1.5ex \@plus .2ex}{\normalfont\bf}}


% Enforce proper line breaks and avoid widows and orphans
% \tolerance 1414
% \hbadness 1414
% \emergencystretch 1.5em
% \hfuzz 0.3pt
% \widowpenalty = 10000
% \clubpenalty=10000
% \vfuzz \hfuzz
% \raggedbottom

%Set date to empty
\date{02/24/16}

% Format footnotes

% Footnote formatting
\usepackage[hang]{footmisc}
\renewcommand{\hangfootparindent}{1em}
\renewcommand{\hangfootparskip}{0em}
\renewcommand{\footnotemargin}{0.00001pt}
\def\footnotelayout{\hspace{1em}}%

%%%%%%%%%%%%%%%%%%%%%%%%%%%%%%%%%%

% Place your own additions to the preable HERE, that is, BEFORE the following
% call to hyperref

%%%%%%%%%%%%%%%%%%%%%%%%%%%%%%%%%%%

%Hyperlink settings for linkcolors and initial view
\usepackage[%
colorlinks=true,%
linkcolor=black,%
citecolor=black,%
urlcolor=EconomicsDarkBlue,%
pdfstartview=FitH,%
pdfview=FitH,%
pdfpagemode=UseNone]{hyperref}

%Define font for hyperlinks
\def\UrlFont{\normalfont}

% Some fine-tuning of layout
%\usepackage{microtype}

%%%%%%%%%%%%%%%%%%%%%%%%%%%%%%%%%%%

% Put your additions to the preable above the call to hyperref, not here, unless you
% want to modify hyperref settings, as is the case with the precvious two commands

% \@ifundefined{showcaptionsetup}{}{%
%  \PassOptionsToPackage{caption=false}{subfig}}
% \usepackage{subfig}
% \AtBeginDocument{
%   \def\labelitemi{\tiny\(\bullet\)}
% }

% \makeatother

\begin{document}

\title{Here is the Title of Your Paper}


\author{Alan Yeung%
\thanks{Give your affiliation and thanks here.%
}.}
\maketitle
\begin{abstract}
\noindent Here comes your abstract.

This is a template for documents prepared with \LaTeX{} for the journal
Economics-The Open Access, Open Assessment E-Journal (www.economics-ejournal.org).

You find further instructions in the Introduction. Please read them.

\noindent \medskip{}


\noindent \emph{Keywords: }Insert the keywords here, separated by
commas.

\noindent \emph{Journal of Economic Literature Classification: }Insert
the classification numbers, such as \ J7, B54,\emph{ }etc. here\emph{.}
\end{abstract}


\newpage{}

\doublespacing

\section{Introduction}\label{intro}

One way to judge the quality of an institution is to use the quality of the student body as a proxy measure. 
Although, there is no standard to measure student quality, there are performance measures that are used to judge the quality of a student body (also referred to as peer quality).
Indeed, for institutions of higher education it is now standard practice to use performance measures of the student body as an indicator of institutional quality.
For example, the average high school GPA, average SAT score, and average class rank are all common reported characteristics of an institutions student body.
The underlying idea of using the quality of the student body as a measure for the quality of the institution, is that the quality of one's peers matter.
Using this idea as our foundation, this paper seeks to answer the question, how does the quality of one's classmates impact a student's grade?
If the quality of one's peers has a strong impact on one's educational experience, than one of the most obvious places for an effect to exist is in the classroom. 

\section{Literature Review}\label{litreview}

There has been a fair amount of recent literature on peer effects in educational institutions.
Specifically, roommate peer effects research has been a trending topic due to the natural quasi-random roommate assignments made by many institutions of higher education.
Quasi-random roommate assignments, such as those used in \citet{griffith2014peer} and \citet{zimmerman2003peer}, allow researchers to avoid the perils of selection bias and lead to theoretically more accurate results.
However, most of the literature is divided on whether or not roommate peer effects exist \citep{griffith2014peer,zimmerman2003peer,sacerdote2000peer,foster2006s,mcewan2006roommate}.
This is a testament to the difficulty of finding peer effects and the difficulty of solving all the econometric problems to ensure that they actually exist.

Much of the literature on peer effects in the classroom exploits quasi-random classroom assignments, in a similar vain to the roommate peer effects literature.
For instance, researchers in \citet{kang2007classroom} exploit the quasi-random classroom assignments in South Korean middle schools, and \citet{carman2012classroom} utilize the fact that students are assigned to Chinese middle schools either randomly or through an admissions test.
Both studies focused on determining the significance of peer effects on classroom grades, and find evidence of peer effects.
\citet{kang2007classroom} finds that strong students have a positive impact on the academic performance other students, while weak students have a negative effect on the performance of other students. 
\citet{carman2012classroom} find somewhat mixed results, but in general they find evidence of positive and significant peer effects on the impact of classmate quality on the academic performance of other students.
Other studies use econometric techniques, such as fixed effects modeling, to correct for any selection bias.\footnote{See \citet{burke2013classroom} as an additional example.}
For example, \citet{schlosser2008inside} and \citet{lavy2012good} control for fixed effects when studying the peer effects of underachievers and overachievers on other students in secondary schools.
Both studies find evidence of peer effects where a larger proportion of lower achieving students in a high school class has a negative impact on the achievement of ``regular'' students in the class.
\citet{lavy2012good} in particular also find a positive peer effect from high achieving peers on girls.
In general, the majority of the literature finds a larger proportion of higher achievers in a class has a positive impact on the academic performance of other students, and a larger proportion of low achievers in a class has a negative impact on the academic performance of other students.\footnote{See the \nameref{results} section for the reasoning.}

We seek to add to the literature by analyzing the peer effects (if any) in higher education, whereas the majority of other studies focus on K-12 education. 
It is possible that the peer dynamics change in the transition from secondary education to higher education, and our data set gives us a unique opportunity to analyze these peer dynamics.
Our data set comes from a small liberal arts institution which should make it an excellent environment to see magnified peer effects.\footnote{For more details see \sectlabel{summarystats}.}

\section{Data}\label{data}

Our data set comes from a selective medium sized liberal arts college in the mid-west, henceforth referred to as the institution, for which we have data on 5 cohorts (2011-2015). 
We use classroom level data in our regressions as it has been shown to generate stronger results than other group level measures \citep{burke2013classroom}. 

In total we have 1,412 observations and use 18 variables. 
Additionally, we use two different types of data in this study. 
The first type is classroom data, that consists of individual student level characteristics as well as classroom characteristics. 
We use data from the first class taken by first year students in order to mitigate selection bias\footnote{Discussed in the \nameref{methods} section.}.
The second type of data we use is point data, which consists of the number of points students bid on classes and their preferences for certain classes. 

\subsection{Classroom Data}\label{classdata}

%We choose to use only data on the first class taken by first year students in order to mitigate selection bias. First year students must select their first course before the semester begins using a point bidding system. Students are accepted into the institution from a variety of areas and backgrounds, and it is very unlikely that students were able to select courses based on any previous relationships. This reduces the likelihood of selection bias effecting our model. In addition, first year students choose courses for a variety of reasons, and these reasons add an element of randomness to the course selection, which also helps to mitigate selection bias. 

The classroom level data on first year students in their first class at the institution is primarily used in the second stage (primary) regression.\footnote{Some classroom level data was also used in the first stage regression. See \sectlabel{pointsdata} for details.}
There are five key variables in our primary regression, Grade, AcadRating (Academic Rating), PctTopX, PctBotX, and PctMidY. 
Grade is the primary outcome of interest, and is the grade received by a student after taking the course. 
Courses are graded on a four point scale, and there are eleven possible grades ranging from an ``A'' (4.0) to ``F'' (0.0).\footnote{The possible grades are A = 4.0, A- = 3.7, B+ = 3.3, B = 3.0, B- = 2.7, C+ = 2.3, C = 2.0, C- = 1.7, D+ = 1.3, D = 1.0, F = 0.0} 
%TODO: Add to citations
As a measure for student ability we follow the literature and use a proxy measure developed on data prior to college enrollment, Academic Rating \citep{griffith2014peer}. 
The Academic Rating is a number assigned to all students at the institution. 
It is a number that represents the culmination of a student's high school GPA, their test scores, the difficulty of the high school curriculum, the quality of their high school, and their writing ability. 
As suggested by the literature these variables are all common (and ``good'') indicators of college academic performance \citep{betts2003determinants,dooley2012persistence}.
The Academic Rating variable ranges from a low of 1 (representing a low ability student) to a high of 65 (representing a high ability student).

We used the Academic Rating variable to create our peer measure variables PctTopX, PctBotX, and PctMidY. 
PctTopX represents the proportion of students in the class that are in the top X percent of the sample based on Academic Rating. 
For instance, PctTop5 represents the proportion of students in the class that are in the top five percent of the sample based on Academic Rating.
PctBotX is defined similarly for the bottom X percentage in terms of Academic Rating. 
PctMidY is the proportion of students in a class that are not in the top X percent or bottom X percent of the sample in terms of Academic Rating. 
The remaining variables, Minority, Female, InState, Intl (International), Needy, ClassSize, URM (Underrepresented Minority), Year, Division, and Professor, were used as control variables in the regression.
Refer to \tablelabel{tab:def1} below for the definitions of all the class level variables used. 

\clearpage{}

\begin{table}[htb]
\centering
\caption{Classroom Variable Definitions}\label{tab:def1}
 \begin{tabular}{|p{0.15\linewidth}|p{0.8\linewidth}|} 
 \hline\hline
 Variable & Definition \\ [0.5ex] 
 \hline\hline
 Grade & The grade received by a student after taking the course. \\ 
 \hline
 AcadRating & Referred to as Academic Rating, a number that represents the culmination of a student's high school GPA, their test scores, the difficulty of the high school curriculum, the quality of their high school, and their writing ability. \\
 \hline
 PctTopX & The proportion of students in the class in the top X percent of the sample based on Academic Rating \\
 \hline
 PctBotX & The proportion of students in the class in the bottom X percent of the sample based on Academic Rating \\
 \hline
 PctMidY & The proportion of students in a class that are not in the top X percent or bottom X percent of the sample in terms of Academic Rating. \\
 \hline
 Minority & A dummy variable representing whether or not the student is non-Caucasian. 1 = non-Caucasian \& 0 = Caucasian\\
 \hline
 Female & A dummy variable representing whether or not the student identifies as a female. 1 = female \& 0 = male\\
 \hline
 InState & A dummy variable indicating whether or not the student is an in-state student.  1 = in-state \& 0 = out of state\\
 \hline
 Intl & A dummy variable indicating whether or not the student is an international student.  1 = international \& 0 = not international\\
 \hline
%TODO: Check this
 Needy & Whether or not a student qualified for need based financial aid.  1 = financial aid \& 0 = no financial aid\\
 \hline
 ClassSize & An integer the represents the total number of students in a class. \\
 \hline
 URM & Under Represented Minority, a dummy variable indicating whether or not the student is non-Caucasian or Asian.  1 = Asian or Caucasian \& 0 = other ethnicity \\
 \hline
 Year & The year the class took place. \\
 \hline
 Division & The subject area of the class, either Natural Science, Social Sciences, or Humanities. \\
 \hline
 Professor & An identifier for the professor teaching the class. \\
[1ex] 
 \hline\hline
\end{tabular}
\end{table}

\clearpage{}

\subsection{Points Data}\label{pointsdata}

At this institution a bidding system is used to ration classes. 
From 2011-2014 the bidding system was as follows: 
students are allotted 20 points and must rank eight classes in terms of their preferences, 
after classes are ranked students then must bid a number between 0 and 20 (inclusive) points per class on their list, 
students with the highest number of points bid per class are allotted seats, and ties are broken randomly. 
If a student does not make it into any class on his or her preference list, then a class is chosen for the student at random. 

In the year 2015, the bidding system was changed in an effort to allow more students to select into a class higher on their preference list. 
The system was changed in the following ways:
students are allotted 100 points and must rank eight classes in terms of their preferences,
after classes are ranked students then must bid a number between 1 and 20 (inclusive) points per class on their list.
The remaining rules from the original system are the same. 
This new system effectively forces students to spread their points into multiple classes (whereas in the original system all points could be placed into one class) and therefore decreases the standard deviation.

The changes to the bidding system affect one of our key variables in the first stage regression, Demand. 
The Demand variable represents the total number of points bid on a course divided by the number of bidders. 
In an attempt to correct for the changes in the bidding system, the 2015 Demand calculations were divided by five, because students received five times the number of points compared to the original system. 
This corrected the mean of Demand in 2015, however the affect on standard deviation still remains. 
Due to the nature of the bidding system changes, we are unable to correct the affect on the standard deviation, but fortunately this does not affect the overall results.\footnote{See the \nameref{results} section for more details.} 

Another key variable in our first stage regression is Ranking. 
Ranking is our dependent variable in the first stage regression, and is a number between one and eight that specifies the student's preference for the course, where one is a high preference, eight is a low preference, and preferences are not repeated. 
We used the remaining variables in the first stage regression, Minority, Female, InState, Intl (International), Needy, AcadRating, URM (Underrepresented Minority), and Subject as control variables. Refer to \tablelabel{tab:def2} below for the definitions of the unique variables used in the first stage regression. 

\begin{table}[htb]
  \centering
  \caption{Points Data Variable Definitions}\label{tab:def2}
  \begin{tabular}{|p{0.15\linewidth}|p{0.8\linewidth}|} 
    \hline\hline
    Variable & Definition \\ [0.5ex] 
    \hline\hline
    Ranking & A number between one and eight specifying the student's preference for the course, where one is a high preference and eight is a low preference. \\ 
    \hline
    Demand & The total number of points bid on a course divided by the number of bidders. For the year 2015, this variable was divided by five to correct for the bidding system changes.\\
    \hline
    %TODO: Create full list
    Subject & The specific subject of the course, such as mathematics, anthropology, chemistry, psychology, etc. For a full list of subjects see Appendix A. \\
    [1ex] 
    \hline\hline
  \end{tabular}
\end{table}

\subsection{Summary Statistics}\label{summarystats}

The descriptive statistics for the non-dummy variables are given below in \tablelabel{tab:summarystats}.\footnote{Control variables, such as Year, Professor, Division, and Subject where not summarized.} 
The outcome of interest, Grade, has a mean that changes slightly over time, while the standard deviation remains fairly constant. 
From 2011-2012 the mean Grade was 3.115, then from 2013-2015 the mean grade increased to 3.15. 
This suggests that their might have been some grade inflation over the years as the mean Academic Rating, a measure of student ability, remained fairly consistent from 2011-2014 (jumping by about 2 points in 2015). 
The variables PctTopX, PctBotX, and PctMidY all vary slightly from their expected values, indicating that the distribution of abilities is not uniform every year. 
That is, one would expect the mean of PctTop5 to always be about 0.05, however this is not the case since the Academic Rating cutoffs are not exact\footnote{Exactly 5\% of students do not have an Academic Rating higher than our cutoff. Instead the number is about 0.048\% and this is true for all of our defined cutoffs.} and the distribution of abilities is not uniform. 
In fact, the distribution of abilities seems to be bias towards recent years as the higher mean Academic Rating, PctTop5, and PctTop10 in 2015 indicate. 
The ClassSize variable jumps from a mean of 11 in 2011-2014 to 14.27 in 2015 because fewer classes were offered and more students were in the incoming class.
The average number of classes offered fell from 29 in 2012-2014 to 25 in 2015.\footnote{In 2011 the mean class size was also 25, however there were fewer students in the incoming class.}
As expected, the Demand variable has a lower standard deviation in 2015 compared to 2011-2014 because the bidding system changes. 

It may be valuable to note that several studies suggest that smaller class sizes have a positive impact on average student achievement (measured by grades and test scores) \citep{diette2015class,kokkelenberg2008effects}. 
The primary reasoning for the inverse relationship between achievement and class size is that students have more quality time to interact with teachers and peers as class size decreases. 
As this institution has relatively small class sizes, we may expect to find magnified peer effects. 

\begin{sidewaystable}
\caption{Summary Statistics}\label{tab:summarystats}
\centering\begin{tabular}{|l|c|c|c|c|c|c|c|c|c|c|}
\hline
\hline
& \multicolumn{2}{|c|}{2011 Data} & \multicolumn{2}{|c|}{2012 Data} & \multicolumn{2}{|c|}{2013 Data} & \multicolumn{2}{|c|}{2014 Data}  & \multicolumn{2}{|c|}{2015 Data}\\
\hline
\prbf{Variable} & \prbf{Mean} & \prbf{Std. Dev.} & \prbf{Mean} & \prbf{Std. Dev.} & \prbf{Mean} & \prbf{Std. Dev.}  & \prbf{Mean} & \prbf{Std. Dev.} & \prbf{Mean} & \prbf{Std. Dev.}\\
\hline
            Grade    & 3.12  & 0.96 & 3.11  & 0.87 & 3.17  & 0.78 & 3.15  & 0.81 & 3.16  & 0.87 \\
            AcadRa{\textasciitilde}g & 50.72 & 6.09 & 50.89 & 5.9  & 49.98 & 6.25 & 50.95 & 6.44 & 52.80 & 6.19 \\
            PctTopQ  & 0.30  & 0.13 & 0.28  & 0.17 & 0.22  & 0.17 & 0.27  & 0.15 & 0.41  & 0.14 \\
            PctTop5  & 0.08  & 0.10 & 0.06  & 0.10 & 0.06  & 0.07 & 0.04  & 0.07 & 0.14  & 0.10 \\
            PctTop10 & 0.16  & 0.12 & 0.14  & 0.13 & 0.14  & 0.11 & 0.16  & 0.13 & 0.26  & 0.13 \\
            PctBotQ  & 0.23  & 0.11 & 0.19  & 0.11 & 0.28  & 0.17 & 0.18  & 0.10 & 0.13  & 0.10 \\
            PctBot5  & 0.04  & 0.06 & 0.05  & 0.08 & 0.06  & 0.12 & 0.06  & 0.08 & 0.04  & 0.06 \\
            PctBot10 & 0.08  & 0.08 & 0.07  & 0.08 & 0.10  & 0.14 & 0.08  & 0.08 & 0.06  & 0.08 \\
            PctMid50 & 0.47  & 0.16 & 0.54  & 0.16 & 0.50  & 0.17 & 0.54  & 0.14 & 0.46  & 0.14 \\
            PctMid90 & 0.87  & 0.11 & 0.89  & 0.12 & 0.87  & 0.14 & 0.90  & 0.12 & 0.82  & 0.10 \\
            PctMid80 & 0.77  & 0.14 & 0.78  & 0.14 & 0.76  & 0.16 & 0.76  & 0.16 & 0.68  & 0.13 \\
            Minority & 0.21  & 0.41 & 0.37  & 0.48 & 0.46  & 0.50 & 0.35  & 0.48 & 0.39  & 0.49 \\
            Female   & 0.47  & 0.50 & 0.55  & 0.50 & 0.53  & 0.50 & 0.51  & 0.50 & 0.54  & 0.50 \\
            InState  & 0.15  & 0.35 & 0.13  & 0.33 & 0.15  & 0.36 & 0.12  & 0.33 & 0.15  & 0.36 \\
            Intl     & 0.02  & 0.15 & 0.01  & 0.08 & 0.06  & 0.23 & 0.05  & 0.22 & 0.07  & 0.25 \\
            Needy    & 0.39  & 0.49 & 0.36  & 0.48 & 0.36  & 0.48 & 0.32  & 0.47 & 0.43  & 0.50 \\
            ClassS{\textasciitilde}e & 10.85 & 3.09 & 11.91 & 3.05 & 11.11 & 2.48 & 11.39 & 2.62 & 14.27 & 2.47 \\
            URM      & 0.11  & 0.31 & 0.29  & 0.45 & 0.36  & 0.48 & 0.27  & 0.44 & 0.30  & 0.46 \\
            Ranking  & 1.77  & 1.54 & 1.70  & 1.44 & 1.79  & 1.41 & 1.91  & 1.79 & 2.14  & 1.41 \\
            Demand   & 2.81  & 0.83 & 2.62  & 0.75 & 2.28  & 0.86 & 2.37  & 0.86 & 2.44  & 0.22 \\
\hline
N & 226 & & 292 & & 290 & & 285 & & 319\\
\hline
\hline
\end{tabular}
\end{sidewaystable}
\newpage{}
\section{Empirical Methodology}\label{methods}

Our foundational peer effects model is as follows:
\setlength{\belowdisplayskip}{6pt} \setlength{\belowdisplayshortskip}{1pt}
\setlength{\abovedisplayskip}{-4pt} \setlength{\abovedisplayshortskip}{1pt}

\begin{equation}\label{eq:0}
G_{i} = \beta_{0} + \beta_{1} Ability_{i} + \beta_{2} Ability_{i}^{CM1} + \beta_{3} Ability_{i}^{CM2} + \overrightarrow{\beta} \overrightarrow{z} + \epsilon_{i}
\end{equation}

This is an OLS model, where $G_{i}$ is the grade received by student $i$ in their first course at the institution, $Ability_{i}$ is a proxy for the student's academic ability (Academic Rating)\footnote{For more information see \sectlabel{data}.}, $Ability_{i}^{CM1}$ and $Ability_{i}^{CM2}$ are classmate ability measures, $\overrightarrow{z}$ is a vector of control variables\footnote{Control variables included Minority, Female, InState, International, Needy, Class Size, URM, Year, Division, and Professor. See \tablelabel{tab:def1} and \tablelabel{tab:def2} for definitions.}, and $\epsilon_{i}$ is the error term. 
$Ability_{i}^{CM1}$ and $Ability_{i}^{CM2}$ are our peer measures and are defined as one of the following, the proportion of high achieving, middle achieving, or low achieving students in a class\footnote{Unfortunately due to data limitations only the proportion of high achievers and low achievers in a class was used for our peer measures, for more details see \sectlabel{results}.}(as defined by cutoffs in academic rating), and $Ability_{i}^{CM1}$ is not the same measure as  $Ability_{i}^{CM2}$. 
Thus $\beta_{2}$ and $\beta_{3}$ are the primary coefficients of interest, as they estimate the impact of the classmate abilities variables (our peer measures) on a student's grade. 
The OLS regression is run on the subset of students that are not part of the peer measures. 
For example, if $Ability_{i}^{CM1}$ and $Ability_{i}^{CM2}$ were defined as the proportion of high achievers in a class and the proportion of low achievers in a class respectively, then the model would be run on the middle achievers. 

However, because our sample is nonrandom\footnote{Unfortunately students are not assigned to classes and institutions randomly, instead the institution selects specific students, students next select the institution, and students then select into classes.} the calculated coefficients of model \eqref{eq:0} are at risk of being bias \citep{heckman1979sample}.\footnote{For more details see \sectlabel{methods:csb}} This bias, known as selection bias, is a serious threat to this type of peer effects model because self selection often exists and leads to a nonrandom sample \citep{carman2012classroom,betts2003determinants,ding2007peers}. Therefore, we must update our model in order to correct for any selection bias that may exist.

\subsection{Two Stage Selection Model}\label{methods:tssmodel}

In an effort to correct for selection bias we use a two stage selection model, similar to the one described in \citet{heckman1979sample}. Our two stage selection model uses an ordered probit model in the first stage and an OLS model in the second stage. From the first stage ordered probit model we take the calculated inverse mills ratios and use them as a control variable in the second stage regressions. The inverse mills ratios are calculated estimates, that when used in the second stage regression, help to control for selection bias \citep{heckman1979sample}.\footnote{For more information see \citet{greene2002limdep}.}

For the first stage regression we use an ordered probit model, defined as follows:

\setlength{\belowdisplayskip}{5pt} \setlength{\belowdisplayshortskip}{1pt}
\setlength{\abovedisplayskip}{-6pt} \setlength{\abovedisplayshortskip}{1pt}

\begin{equation}\label{eq:1}
R_{i}^{*} = \alpha_{1} D_{i} + \overrightarrow{\alpha} \overrightarrow{\omega} + \epsilon_{i}
\end{equation}

\setlength{\belowdisplayskip}{11pt} \setlength{\belowdisplayshortskip}{1pt}
\setlength{\abovedisplayskip}{-4pt} \setlength{\abovedisplayshortskip}{1pt}

\begin{equation}\label{eq:2}
R_{i} = 
\begin{cases}
  1 \ \ if \ \ - \infty < R_{i}^{*} \leq \mu_{1} \\
  \vdots \\
  j \ \ if \ \ \mu_{j} < R_{i}^{*} < \infty
\end{cases}
\end{equation}

Where $D_{i}$ is the demand for the student's selected class, $\overrightarrow{\omega}$ is a vector of control variables\footnote{Specifically the variables include Minority, Female, InState, Intl, Needy, AcadRating, URM (Underrepresented Minority), and Subject. See \tablelabel{tab:def1} and \tablelabel{tab:def2} for definitions.}, and $\epsilon_{i}$ is the error term. 
In \eqref{eq:2} we see that the unobserved selection variable $R_{i}^{*}$ corresponds to the observed $R_{i}$ through $\mu$, a vector of unknown cutoffs. 
The variable $j$ represents the number of selection categories there are, where in any of our regressions $j$ is at least two and at most four. 
For example, if $j$ is equal to two, then $R_{i}$ equal to one represents those students who selected into their first choice class and $R_{i}$ equal to two represents those students who did not select into their first choice class (instead selecting into their second choice, third choice, etc.). 
The ordered probit model ultimately estimates the probability that $R_{i}$ is equal to $j$ using $R_{i}^{*}$, that is, 

\setlength{\belowdisplayskip}{5pt} \setlength{\belowdisplayshortskip}{1pt}
\setlength{\abovedisplayskip}{-4pt} \setlength{\abovedisplayshortskip}{1pt}

\begin{equation}\label{eq:3}
Pr(R_{i} = j) \ = \ Pr(\mu_{j-1} < R_{i}^{*} \leq \mu_{j})
\end{equation}

\noindent Once we calculate the first stage ordered probit, we use the inverse mills ratios in the second stage OLS regression.

We use an OLS regression very similar to our foundational model for our second stage model, the only difference is that the inverse mills ratios are added as a control variable to correct for selection bias. The model is as follows:

\begin{equation}\label{eq:4}
G_{i} = \beta_{0} + \beta_{1} Ability_{i} + \beta_{2} Ability_{i}^{CM1} + \beta_{3} Ability_{i}^{CM2} + \beta_{4} \lambda_{i} + \overrightarrow{\beta} \overrightarrow{z} + \epsilon_{i}
\end{equation}

Where the variables are defined in the same manner as \eqref{eq:0}, and $\lambda_{i}$ is the inverse mills ratios calculated in the first stage regression.
Just as in \eqref{eq:0} the OLS regression is run on the subset of students that are not part of the peer measures. 
However, the calculation of different inverse mills ratios for each of the possible selection categories makes it necessary for a separate regression to be run on each of the students who selected into a particular selection category.\footnote{This is why there are Grade1, Grade2, etc. categories in the regressions seen in \sectlabel{results}.}
As an example, suppose $Ability_{i}^{CM1}$ and $Ability_{i}^{CM2}$ were defined as the proportion of high achievers in a class and the proportion of low achievers in a class respectively and the number of selection categories ($j$) is equal to two. 
The second stage OLS model will first be regressed on the middle achievers who selected into their first choice class, then on the middle achievers who did not select into their first choice class, producing two sets of regression outputs. 

\subsection{Controlling Selection Bias}\label{methods:csb}

By using the ranking and points data in the two stage selection model we are attempting to correct for selection bias, a serious problem in many peer effect models \citep{carman2012classroom,betts2003determinants,ding2007peers}. 
Specifically, our model is controlling for any selection bias that results from students selecting into classes high on their preference list. 
The idea is that there are unobservable factors that correlate with both student class preferences and the grade received in the class, which would then bias our coefficients of interest ($\beta_{2}$ and $\beta_{3}$). 
For instance, it may be that popular courses draw a higher proportion of high achieving students, while the majority of middle achieving students are drawn into those same courses because they enjoy the subject. 
In this case a higher proportion of high achievers is not causal to the increased grades of middle achievers. 
Instead, an unobservable variable (passion for the subject) drives middle achievers to self select into courses where they will achieve higher grades and there happen to be a higher proportion of high achievers. 
In such a case, the self-selection of middle achieving students would bias our coefficients of interest and our results would be inaccurate.
By using the two stage regression model, we are attempting to correct for this type of selection bias.

\section{Results}\label{results}

The results suggest that the proportion of high achievers in a class has a significant, but small, negative impact on the grades of middle achievers, while the proportion of low achievers in a class has a significant, but small, positive impact on the grades of middle achievers\footnote{See \tablelabel{tab:2sc}, \tablelabel{tab:3sc}, and \tablelabel{tab:4sc}}.
We find that the estimated coefficients are too small to have an economically significant impact on final student grades, however if these trends continue in the long run they may impact cumulative GPA. 
These results are contradictory to those found in the majority of the literature and we discuss some possible explanations for this below \citep{kang2007classroom,carman2012classroom,burke2013classroom,schlosser2008inside,lavy2012good}. 
We begin the section by evaluating our results in detail and describing the patterns found.
Then, we conclude by discussing the implications of our results and provide some suggestions for further research. 

%It should be noted that even though we attempt to correct for any selection bias that may exist, our results should still be interpreted with caution, as there may exist model misspecifications that still make our model susceptible to selection bias. 

It is important to note that the main results of this paper focus on the impact of high achievers and low achievers on the grades of middle achievers. 
Unfortunately, data limitations made it difficult to measure the impact of classmate ability on high achievers or low achievers.\footnote{Specifically we could not measure the impact of the proportion of low achievers and middle achievers on high achievers and similarly for low achievers.} 
There are not enough, observations at this time, within either group (high achievers and low achievers) to run the two stage regression model. 
However, the regressions run on the middle achievers still make an important contribution as the majority of students are middle achievers, so our findings impact most of the population.

\tablelabel{tab:2sc}, \tablelabel{tab:3sc}, and \tablelabel{tab:4sc} show the relevant results\footnote{The results of the regression for all the variables can be found in \appendixlabel{appendix:b}.} of the two stage selection model (outlined in section~\ref{methods}) run on the middle achievers, where the selection equation \eqref{eq:1} is a two category ordered probit, three category ordered probit, and four category ordered probit respectively. 
Each table contains the relevant results from three regressions where the major difference between each regression within a table is the measure of the ``peer'' used. 
For the regressions found in \tablelabel{tab:2sc}, \ref{tab:3sc}, and \ref{tab:4sc}, we use the proportion of high achievers and low achievers in a class as a peer measure and regress on middle achieving students. 
After each regression, the measures used for a high and low achieving peer become stricter (and therefore the measure for the middle achievers becomes less strict). 
For instance, regression 1 (labeled Top/Bottom 25\%) in \tablelabel{tab:2sc} focuses on the impact of the proportion of high achievers, defined as the twenty-five percent with the highest academic rating in the sample, and the proportion of low achievers, the twenty-five percent with the lowest academic rating in the sample, on the grades of the middle achievers (those that are not high or low achievers). 
In each regression, GradeX refers to the regression run on the middle achievers that selected into their X choice class based on the first stage ordered probit.\footnote{For more details see \sectlabel{methods:tssmodel}} 

\clearpage{}

\begin{table}[htb]
  \centering
  \begin{threeparttable}
    \caption{Two Selection Categories}\label{tab:2sc}
    \begin{tabular}{l l l l} 
      \hline
      \hline
               & Top/Bott{\textasciitilde}25\% & Top/Bott{\textasciitilde}10\% & Top/Bott{\textasciitilde}5\% \\
               & (Std. Err.)                   & (Std. Err.)                   & (Std. Err.)                  \\
      \hline
      Grade1   &                               &                               &                              \\
      PctBotQ  & -0.150                        &                               &                              \\
               & (0.53)                        &                               &                              \\
      PctTopQ  & -0.787**                      &                               &                              \\
               & (0.41)                        &                               &                              \\
      PctBot10 &                               & 0.612                         &                              \\
               &                               & (0.54)                        &                              \\
      PctTop10 &                               & -0.741***                     &                              \\
               &                               & (0.33)                        &                              \\
      PctBot5  &                               &                               & 1.001**                      \\
               &                               &                               & (0.60)                       \\
      PctTop5  &                               &                               & -0.702**                     \\
               &                               &                               & (0.41)                       \\
      \hline
      Grade2   &                               &                               &                              \\
      PctBotQ  & 1.720**                       &                               &                              \\
               & (1.00)                        &                               &                              \\
      PctTopQ  & 0.660                         &                               &                              \\
               & (0.70)                        &                               &                              \\
      PctBot10 &                               & 0.370                         &                              \\
               &                               & (0.78)                        &                              \\
      PctTop10 &                               & 0.827**                       &                              \\
               &                               & (0.49)                        &                              \\
      PctBot5  &                               &                               & -0.061                       \\
               &                               &                               & (0.70)                       \\
      PctTop5  &                               &                               & 0.198                        \\
               &                               &                               & 0.198                        \\
      \hline
      \hline
    \end{tabular}
    \begin{tablenotes}
    \item{* p<.2, ** p<.1, *** p<.05 \\Note: Regressions run using 3 different peer measures and a two choice ordered probit for the first stage.}
    \end{tablenotes}
  \end{threeparttable}
\end{table}

\clearpage{}

%\thispagestyle{empty}
\begin{table}[htb]
  \centering
  \begin{threeparttable}
    \caption{Three Selection Categories}\label{tab:3sc}
    \begin{tabular}{l l l l} 
      \hline
      \hline
               & Top/Bott{\textasciitilde}25\% & Top/Bott{\textasciitilde}10\% & Top/Bott{\textasciitilde}5\% \\
               & (Std. Err.)                   & (Std. Err.)                   & (Std. Err.)                  \\
      \hline
      Grade1   &                               &                               &                              \\
      PctBotQ  & -0.314                        &                               &                              \\
               & (0.52)                        &                               &                              \\
      PctTopQ  & -0.840***                     &                               &                              \\
               & (0.39)                        &                               &                              \\
      PctBot10 &                               & 0.602                         &                              \\
               &                               & (0.54)                        &                              \\
      PctTop10 &                               & -0.705***                     &                              \\
               &                               & (0.33)                        &                              \\
      PctBot5  &                               &                               & 1.008**                      \\
               &                               &                               & (0.60)                       \\
      PctTop5  &                               &                               & -0.667*                      \\
               &                               &                               & (0.41)                       \\
      \hline
      Grade2   &                               &                               &                              \\
      PctBotQ  & 1.248                         &                               &                              \\
               & (1.41)                        &                               &                              \\
      PctTopQ  & 1.614**                       &                               &                              \\
               & (0.92)                        &                               &                              \\
      PctBot10 &                               & -0.138                        &                              \\
               &                               & (1.02)                        &                              \\
      PctTop10 &                               & 0.439                         &                              \\
               &                               & (0.70)                        &                              \\
      PctBot5  &                               &                               & -0.819                       \\
               &                               &                               & (0.88)                       \\
      PctTop5  &                               &                               & -0.585                       \\
               &                               &                               & (1.00)                       \\
      \hline
      Grade3   &                               &                               &                              \\
      PctBotQ  & 1.215                         &                               &                              \\
               & (1.54)                        &                               &                              \\
      PctTopQ  & -0.961                        &                               &                              \\
               & (1.16)                        &                               &                              \\
      PctBot10 &                               & 0.002                         &                              \\
               &                               & (1.18)                        &                              \\
      PctTop10 &                               & 0.541                         &                              \\
               &                               & (0.63)                        &                              \\
      PctBot5  &                               &                               & -0.170                       \\
               &                               &                               & (1.13)                       \\
      PctTop5  &                               &                               & -0.031                       \\
               &                               &                               & (0.83)                       \\
      \hline
      \hline
    \end{tabular}
    \begin{tablenotes}
    \item{* p<.2, ** p<.1, *** p<.05 \\Note: Regressions run using 3 different peer measures and a three choice ordered probit for the first stage.}
    \end{tablenotes}
    \centering
%    \thepage
  \end{threeparttable}
\end{table}

\clearpage{}

\begin{sidewaystable}[htb]
  \centering
  \begin{threeparttable}
    \caption{Four Selection Categories}\label{tab:4sc}
    \def\arraystretch{1.5}
    \begin{tabular}{l|c|c|c|c|c|c|c|c} 
      \hline
      \hline
      
      & \multicolumn{2}{|c|}{Grade1} & \multicolumn{2}{|c|}{Grade2} & \multicolumn{2}{|c|}{Grade3} & \multicolumn{2}{|c}{Grade4} \\
      \hline
      & \prbf{PctBot5} & \prbf{PctTop5} & \prbf{PctBot5} & \prbf{PctTop5} & \prbf{PctBot5} & \prbf{PctTop5} & \prbf{PctBot5} & \prbf{PctTop5} \\
      \hline
      Top/Bott{\textasciitilde}5\% & 0.998** & -0.669* & -0.791 & -0.530 & 2.443* & -0.215 & -0.417 & -2.847*** \\
      (Std. Err.) &(0.61) &(0.41) &(0.88) &(1.01) &(1.64) &(1.52) &(1.72) &(1.18) \\

      %         & \Top/Bott{\textasciitilde}5\% \\
      %         & (Std. Err.)                  \\
      % \hline
      % Grade1  &                              \\
      % PctBot5 & 0.998**                      \\
      %         & (0.61)                       \\
      % PctTop5 & -0.669*                      \\
      %         & (0.41)                       \\
      % \hline
      % Grade2  &                              \\
      % PctBot5 & -0.791                       \\
      %         & (0.88)                       \\
      % PctTop5 & -0.530                       \\
      %         & (1.01)                       \\
      % \hline
      % Grade3  &                              \\
      % PctBot5 & 2.443*                       \\
      %         & (1.64)                       \\
      % PctTop5 & -0.215                       \\
      %         & (1.52)                       \\
      % \hline
      % Grade4  &                              \\
      % PctBot5 & -0.417                       \\
      %         & (1.72)                       \\
      % PctTop5 & -2.847***                    \\
      %         & (1.18)                       \\
      \hline
      \hline
    \end{tabular}
    \begin{tablenotes}
    \item{* p<.2, ** p<.1, *** p<.05 \\Note: Regressions run using 3 different peer measures and a three choice ordered probit for the first stage. Additionally, Top/Bottom 25\% and Top/Bottom 10\% regressions could not be run because there is too little variance in the data at those levels (most likely because there are too few observations).}
    \end{tablenotes}
  \end{threeparttable}
\end{sidewaystable}

\clearpage{}

When analyzing the results in \tablelabel{tab:2sc}, \ref{tab:3sc}, and \ref{tab:4sc} there are a few important patterns to note. 
First, the majority of the statistically significant results are found in students who selected into their first choice (Grade1), and students who selected into their second choice or below (Grade2, Grade3, etc.) show inconsistent results. 
Next, all the significant results in Grade1 are consistent, in that the proportion of high achievers has a negative impact on the grades of middle achievers and the proportion of low achievers has a positive impact on the grades of middle achievers. 
Furthermore, the majority of significant results appear as the peer measure becomes stricter, that is more significant results appear as the peer measure changes from the top/bottom twenty-five percent, to ten percent, and finally to five percent. 
 
The Grade1 results in \tablelabel{tab:2sc}, \ref{tab:3sc}, and \ref{tab:4sc} show a consistent pattern. 
The proportion of high achievers in a classroom has a negative impact on the grade of middle achievers, regardless of the strictness of the peer measure. 
As the peer measure becomes stricter (specifically at the five percent threshold), we see that the proportion of low achievers in a classroom has a positive impact on the grades of middle achievers. 
There is some precedence for using the five percent threshold and finding peer effects. 
In \citet{lavy2012good} researchers also found strong peer effects using the top and bottom five percent as their peer measures. 
\citeauthor{lavy2012good} argue that their use of the top and bottom five percent is not arbitrary by showing that it is precisely those students around the five percent threshold that have a strong peer impact on fellow students. 
That is, they showed that students in the middle ninety percent of the ability distribution do not show strong peer effects of any sort, while the top and bottom five percent are the students who are the most influential. 

Our results run counter to those found in the current literature\footnote{\citet{burke2013classroom} do find one result similar to ours, but they do not provide any explaination for their finding.}, granted that the current literature is rather scarce especially in the area of higher education \citep{kang2007classroom,carman2012classroom,burke2013classroom,schlosser2008inside,lavy2012good}. 
Most of the current literature finds that low achievers hurt the grades of middle achieving students and high achievers help the grades of middle achieving students. 
Similar results have also been discovered in roommate peer effects literature \citep{griffith2014peer,zimmerman2003peer,sacerdote2000peer}.
The reasoning being that low achieving students may disrupt learning (possibly by instilling bad study habits in the middle achieving students) and the high achieving students may facilitate learning (possibly by asking more relevant questions or helping to tutor middle achieving students). 
This reasoning is very compelling and is exactly what many people would expect to happen, which makes our results of the contrary all the more puzzling.  

We have developed two possible explanations for these results. 
One possibility is that since most of the previous literature does not focus on courses in higher education, peer dynamics have changed in the transition from K-12 education to undergraduate education. 
Most of the literature focuses on K-12, therefore, it is reasonable to suppose that peer effects at institutions of higher education may differ from those seen in K-12 education. 
It is possible that high achieving students in college have a negative peer effect on grades, perhaps through demoralizing other students, and low achieving students have a positive peer effect on grades, perhaps through increasing the number of group study sessions. 
Another possibility is that grade curving is occurring in many of the classes, and is overpowering any actual peer effects. 
That is to say, a student's grade in a class may be determined by her relative performance to her classmates instead of by her absolute performance. 
Unfortunately, due to the time span of the data and faculty changes, it is not feasible to uncover which classes were truly graded on a curve (in which case these findings would not be surprising) and which classes were not. 
Furthermore, if grade curving is the underlying cause of the results we are finding, then it is still possible that peer effects that are in line with the literature exist, but the curving effects are simply overwhelming any peer effects that are occurring. 

Next, we'll explore the possible reasons behind the inconsistent results in the regressions run on students who selected into a class that was not their first choice. 
One reason, may be that for students who do not select into their first choice class, the peer effect dynamics change and aren't as strong. 
Perhaps the students who did not select into their first choice class have less motivation (or something of the sort) to perform in their non-first choice class, and this leads to less peer interactions (or interactions of a different kind) that ultimately lead to smaller/immeasurable peer effects. 
Another explanation, one that we believe is more likely, is that there are simply fewer students who selected into a course that is not their first choice and the results are therefore less reliable. 
In the best cases the number of students who did not select into their first choice is about one half of the number of students who did select into their first choice.\footnote{See \tablelabel{tab:freq_Ranking} in \sectlabel{data:pointsdata} for more details} 
That is, for a two option ordered probit selection model (used in \tablelabel{tab:2sc}), the number of students who are regressed in the Grade1 calculations (those who selected into their first choice class) is twice that of the number of students are regressed in the Grade2 calculations (those who did not select into their first choice class).\footnote{For more details see \sectlabel{methods}.} 
Additionally, as the number of options in the selection model increases (as is the case in \tablelabel{tab:3sc} and \ref{tab:4sc}) the number of students who did not select into their first choice class is further divided into those who selected into their second choice class, third choice class, etc. and regressed on. 
This essentially means that the regressions run on the students who did not select into their first choice class (Grade2, Grade3, Grade4) have far fewer observations than the regressions run on students who did select into their first choice class (Grade1). 
In this case, fewer observations effectively means a smaller sample size, and studies that use smaller sample sizes find that the sample size may lead to insignificant or inaccurate results \citep{gonzalez2013gibrat} and the results of these studies are often not generalizable \citep{oladipupo2013does}.
This is a troubling problem as it suggests that the results from the Grade2, Grade3, and Grade4 regressions are more unreliable than the results of the Grade1 regressions. 
However, the only way to resolve this problem is to collect several more years of data, which is not feasible at this time. 
Therefore, we will focus on the Grade1 regression results for the remainder of the section, and since the majority of students selected into their first choice class, they are arguably the most relevant results regardless. 

Additionally, our results are somewhat imbalanced. That is to say, high achieving students clearly have a negative impact on middle achieving students throughout the majority of our regressions, but low achieving students only have a statistically significant positive impact on middle achieving students when the peer measure becomes stricter. 
One explanation may be that it is pure statistical chance that high achievers had a significant negative affect more often than low achievers had a significant positive affect. 
Our sample size of 903 students who selected into their first choice class is not as large as some found in the literature, which reach into the many thousands \citep{kang2007classroom,lavy2012good}. 
Another possibility is that top students are very effective at having a negative impact on other student's grades, and this result is simply robust to changes in the peer measure. 
Finally, if grade curving is the underlying cause of our results, it may be that high achieving students dull out a middle achieving student's performance more than a low achieving student helps a middle achieving student shine.
Since professors may curve grades in a subjective fashion, they may be affected by this psychological phenomenon. 
Currently we do not have an explanation for this phenomena, although there is evidence of similar kinds of psychological imbalance in other areas. 
For example take the well documented theory of loss aversion. 
As explained by \citet{tversky1991loss}, loss aversion implies that people experience greater impact from a loss when compared to a gain equal in magnitude, and this phenomena has been found in many different areas of human behavior \citep{shalev2002loss,goette2004loss}. 
It is possible that the same psychological mechanisms driving the imbalance in losses and gains (loss aversion) are also driving the imbalance in the effect of the proportion of high achievers and low achievers on middle achievers grades. 

There are two concerns with the methodology used in this paper that need to be addressed.
One concern is selection bias and the other is the change in 2015 Demand\footnote{See \sectlabel{data:pointsdata} for more details.} which may affect the model and overall results.
In order to correct for selection bias we use the two stage selection model outlined in \sectlabel{methods:tssmodel}, and after comparing our results to the foundational OLS model outlined in \eqref{eq:0}, the overall results are the same.\footnote{See \appendixlabel{appendix:c} for the regular OLS results.} 
This suggests that selection bias is most likely not affecting our sample to any significant degree, but selection bias may still be a concern with any future studies that utilize a different sample.
To address our second concern, we correct Demand in 2015\footnote{See \sectlabel{data:pointsdata} for more details on the correction methods we used.} to eliminate some of the effects of the changes in the bidding system, however some effects persisted\footnote{In particular the standard deviation of Demand in 2015 was still affected. See \sectlabel{data} for more details.} and may have affected our results.
In order to verify the robustness of our results we reran the regressions on the same data from the years 2011-2014, with the original bidding system, and found that the same overall results persist.\footnote{TODO: See appendix X for the full results.}
Therefore neither of our initial concerns had a significant affect on the overall results of the paper.

\subsection{Discussion}\label{results:discussion}

The above results indicate the presence of an interesting trend at this institution, but are there any significant implications from these trends?
 In order to shed some light on the implications of our results, we must analyze the magnitudes of the estimated coefficients. 
The average class size in our data set is twelve students (which has remained fairly constant throughout the time period of our data set), and if we define high achievers as the top five percent of students then there is one student per class that is a high achiever,  assuming a uniform distribution of high achievers. 
At the five percent threshold, our results indicate that for every ten percent increase in the proportion of high achievers in the classroom the GPA of a middle achiever falls by roughly 0.07 points. 
In order for the proportion of high achievers in an average class to be high enough to have a significant\footnote{Defined as at least changing the sign of a student's grade which takes about 0.3 grade points.} influence on the grades of middle achieving students, forty percent of the average class (five students) must be high achievers.\footnote{0.4 * 0.7 = 0.28} Using the five percent threshold, there are two classes (about 1.5\%) in our entire data set which have a proportion of high achievers of at least forty percent. 
It is clear that having a class consist of at least forty percent high achievers is an unlikely scenario, and should not be a practical concern for most students. 
Similar calculations can be done to show that having a high enough number of low achieving students in a classroom to significantly affect grades is also quite uncommon. 
However, two considerations must be taken into account. 
First, if these results continued in further classes, then the cumulative effect on the GPA may be significant. 
Depending on a student's degree path, they may be more likely to select into classes with a higher proportion of low achieving students or classes with a higher proportion of high achieving students. 
Additionaly, higher level classes at the institution may number under ten students (depending on the popularity of the degree), therefore if these trends continue it is possible for the cumulative GPA (as well as class grades) of a student to be affected. 
This may be a serious problem in unpopular degrees with abnormally small class sizes. 
The second consideration is that the coefficients are estimates, and it may be that the real coefficients are much larger or smaller than those estimated. 
If the real coefficients happen to be larger, then the results may indeed be a practical concern to students and the administration.

The calculations above show that even though these trends exist, in the short run, the impact on final grades in the first course is most likely not a practical concern to most students. 
However, more research need to be conducted to determine the long run implications of these results. 

There are two types of policies that the majority of the literature discusses, ``streaming'' and ``mixing'' \citep{ding2007peers,kang2007classroom,carman2012classroom}. 
Streaming refers to the policy of grouping students of similar ability levels, while mixing generally refers to the policy of combining students of different ability levels \citep{ding2007peers}.
However, in this case it is difficult to suggest any policy implications without restricting student course choices and without knowing how high achievers and low achievers are affected by classmates of different ability levels. 

As \citet{burke2013classroom}, \citet{carman2012classroom}, and \citet{ding2007peers} find, an optimal policy prescription may not exist.
Suppose that the grades of high achievers benefit from middle and low achievers, while the grades of low achievers are harmed by middle and high achievers,\footnote{Which is what \appendixlabel{appendix:c} suggests to be the case.} then a trade off will need to be made if a policy is to be implemented. 
In this case, there is no grouping which would lead to a positive impact on all student grades. 
For instance, if middle achievers are prioritized, since there are more of them, then they should be placed into classes with as many low achievers as possible. 
However, this leads to two problems.
First, the grades of low achievers would be harmed in this case.
Additionally only a select group of middle achievers could benefit from this policy, since there are far fewer low achievers. 
Also, if too many middle achievers are placed into a class, the proportion of low achievers would be diluted to a point where the affects are insignificant. 
If, on the other hand, the grades of low achievers are positively impacted by the proportion of middle achievers in the class, then by grouping low achievers and middle achievers the grades of both may be maximized.
Again, only a select group of middle achievers would benefit, but this would still maximize the average grade received by all groups overall.
However, to effectively create this optimal grouping, the institution would need to decide to restrict student course choices based on their ability levels, which may not be a popular decision among current or prospective students.\footnote{Institutions could also incentivise students of certain ability levels to take certain courses, however this may be considered unfair to those students who are not incentivised.}

This study is somewhat limited by the sample size. 
With more observations we may be able to determine how high achievers and low achievers are affected by each other and middle achievers.
Additionally, it is an open question as to whether or not these results may be generalized to other types of institutions. 
This study was conducted on data from a small private liberal arts college and the results may differ at different types of institutions with potentially different peer interactions (such as at large universities). 
Also, due to the time span of the data, we are unable to determine whether peer effects, grade curving, or both are responsible for our results. 

Fortunately, these limitations may lead to promising areas of future research. 
In a few years there may be enough observations to rerun the regressions on high and low achievers and discover how they are affected by students of other ability levels in the class. 
With this full picture of classroom peer effects one might find unexpected peer effects (such as if low achievers benefited from high achievers and high achievers benefited from low achievers) that are currently unexplained and have clearer policy implications.
Another avenue for further research, would be to conduct similar regressions at different types of institutions.
It would be interesting to know if the same peer effects are found at community colleges or large universities. 
Both community colleges and large universities are a fundamentally different type of institution compared to a small private liberal arts school, and it is possible that different types (or the same types) of peer effects exist. 
Additionally, different liberal arts colleges may have different grading policies (i.e. a specific policy on grade curving), and by comparing the results from another (similar) institution to this one it may help determine if grade curving is an underlying cause of our results. 
Finally, it may be interesting to analyze classroom peer effects based on gender composition. 
\citet{oosterbeek2014gender} study gender peer effects in a university and find that ``males, but not females, perform poorer in courses with a high math component if the share of females in their work group increases.'' 
It is possible that the majority of our results are due to gender peer effects (instead of the ability peer effects that we attribute them to). 
That is, since high achievers are generally female (71\% are female at the 5\% threshold) the strongest driver of our overall results may be from the impact of a high achieving female on the grade of a middle achieving male. 



\section{Conclusion}\label{Conclusion}
             
\newpage{}

\singlespacing

\bibliography{references}

\end{document}
