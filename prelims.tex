\title{Peer Effects in the Classroom: The Impact of Classmate Abilities on Student Grades in Higher Education}

\author{Alan Yeung}

% \begin{titlepage}
%   \centering
%   \vspace*{-.25cm}
%   {\Large \thetitle \par}
%   \vspace{1.5cm}
%   \begin{center}
%     \line(1,0){250}
%   \end{center}
%   \vspace{1.5cm}
%   {A THESIS \par}
%   \vspace{.25cm}
%   {Presented to \par}
%   \vspace{.25cm}
%   {The Faculty of the Department of Economics and Business \par}
%   \vspace{.25cm}
%   {The Colorado College \par}
%   \vspace{3cm}
%   {In Partial Fulfillment of the Requirements for the Degree \par}
%   \vspace{.25cm}
%   {Bachelor of Arts \par}
%   \vspace{.75cm}
%   {By \par}
%   \vspace{.25cm}
%   {\theauthor \par}
%   \vspace{.25cm}
%   {\thedate \par}
%   \vfill
% \end{titlepage}

\clearpage{}
% Abstract page

\begin{center}
  \Large \thetitle \par
  \vspace{.25cm}
  \large \theauthor \par
  \vspace{.25cm}
  \large \thedate \par
  \vspace{.25cm}
  \large Economics
\end{center}

\begin{abstract}
  \noindent 
Peer effects undoubtedly play an important role in educational attainment and development. 
We investigate the role of peer effects on classroom academic performance at an institution of higher education. 
We use data from a small private liberal arts college and measures of classmate ability levels to estimate a two stage selection model, and find that the proportion of high achievers in a class has a consistent significant negative impact on the grades of middle achieving students. 
Additionally, we find evidence of a significant positive impact on the grades of middle achieving students from the proportion of low achievers in a class. 
The effect of the proportion of high achievers on the grades of middle achievers is economically significant, whereas the effect of the proportion of low achievers is economically insignificant. 
Our results are limited by the size of the data set, and it is unclear how well these results generalize, as the current research on this subject is scarce.

\bigskip

\noindent \emph{Keywords: }Peer Effects, Higher Education, Selection Model

  % \noindent \emph{Journal of Economic Literature Classification: }Insert
  % the classification numbers, such as \ J7, B54,\emph{ }etc. here\emph{.}
\end{abstract}

% \newpage{}
% % Honor code page

% \vspace*{7cm}
% \begin{changemargin}{1.75cm}{1.75cm}
%   ON MY HONOR I HAVE NEITHER GIVEN NOR RECEIVED UNAUTHORIZED AID ON THIS THESIS
% \end{changemargin}
% \vspace{7cm}
% \begin{changemargin}{9cm}{1.75cm}
%   \begin{center}
%     \line(1,0){175}
%   \end{center}
%   \vspace{-.5cm}
%   Signature
% \end{changemargin}

% \newpage{}
% % Acknowledgements page

% \begin{center}
%   \LARGE ACKNOWLEDGMENTS
% \end{center}

% I'd like to thank my thesis advisor, Pedro de Araujo, for all of the advice, support, and wisdom he has provided me throughout this process. I'd also like to thank Kevin Rask for his help in the data acquisition process and all of his quick responses to my many questions and requests. Without these two, this thesis would not have been possible. 

% \newpage{}
% % Table of Contents page

% \thispagestyle{plain}
% \tableofcontents

\newpage
