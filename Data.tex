\documentclass[12pt,a4paper,english,fleqn]{article}
\usepackage{mathptmx}
\usepackage[scaled=0.86]{helvet}
\renewcommand{\familydefault}{\rmdefault}
\usepackage[T1]{fontenc}
\usepackage[latin9]{inputenc}
\usepackage{fancyhdr}
\pagestyle{fancy}
\setcounter{secnumdepth}{2}
\usepackage{setspace}
\usepackage[authoryear,round]{natbib}
\bibliographystyle{agsm}
\usepackage{babel}
\usepackage{chngcntr}
\counterwithin{table}{section}

\renewcommand\headrulewidth{0pt}
\renewcommand\headheight{17pt}

% Caption formatting
\usepackage[font=footnotesize,labelfont=bf]{caption}


% Reduce font size for section
\makeatletter
\renewcommand\section{\@startsection{section}{1}{\z@}
{-5ex \@plus -1ex \@minus -.2ex}{2.3ex \@plus.2ex}{\normalfont\large\bf}}

%Reduce font size for subsection
\renewcommand\subsection{\@startsection{subsection}{2}
{\z@}{-3.25ex\@plus -1ex \@minus -.2ex}{1.5ex \@plus .2ex}{\normalfont\bf}}


% Enforce proper line breaks and avoid widows and orphans
% \tolerance 1414
% \hbadness 1414
% \emergencystretch 1.5em
% \hfuzz 0.3pt
% \widowpenalty = 10000
% \clubpenalty=10000
% \vfuzz \hfuzz
% \raggedbottom

%Set date to empty
\date{02/24/16}

% Format footnotes

% Footnote formatting
\usepackage[hang]{footmisc}
\renewcommand{\hangfootparindent}{1em}
\renewcommand{\hangfootparskip}{0em}
\renewcommand{\footnotemargin}{0.00001pt}
\def\footnotelayout{\hspace{1em}}%

%%%%%%%%%%%%%%%%%%%%%%%%%%%%%%%%%%

% Place your own additions to the preable HERE, that is, BEFORE the following
% call to hyperref

%%%%%%%%%%%%%%%%%%%%%%%%%%%%%%%%%%%

%Hyperlink settings for linkcolors and initial view
\usepackage[%
colorlinks=true,%
linkcolor=black,%
citecolor=black,%
urlcolor=EconomicsDarkBlue,%
pdfstartview=FitH,%
pdfview=FitH,%
pdfpagemode=UseNone]{hyperref}

%Define font for hyperlinks
\def\UrlFont{\normalfont}

% Some fine-tuning of layout
%\usepackage{microtype}

%%%%%%%%%%%%%%%%%%%%%%%%%%%%%%%%%%%

% Put your additions to the preable above the call to hyperref, not here, unless you
% want to modify hyperref settings, as is the case with the precvious two commands

% \@ifundefined{showcaptionsetup}{}{%
%  \PassOptionsToPackage{caption=false}{subfig}}
% \usepackage{subfig}
% \AtBeginDocument{
%   \def\labelitemi{\tiny\(\bullet\)}
% }

% \makeatother

\begin{document}

\title{Here is the Title of Your Paper}


\author{Alan Yeung%
\thanks{Give your affiliation and thanks here.%
}.}
\maketitle
\begin{abstract}
\noindent Here comes your abstract.

This is a template for documents prepared with \LaTeX{} for the journal
Economics-The Open Access, Open Assessment E-Journal (www.economics-ejournal.org).

You find further instructions in the Introduction. Please read them.

\noindent \medskip{}


\noindent \emph{Keywords: }Insert the keywords here, separated by
commas.

\noindent \emph{Journal of Economic Literature Classification: }Insert
the classification numbers, such as \ J7, B54,\emph{ }etc. here\emph{.}
\end{abstract}
\newpage{}

\doublespacing

\section{Data}

We use data from a selective medium sized liberal arts college in the mid-west, henceforth referred to as the institution, for which we have data on 5 cohorts (2011-2015). Classroom level data is used in our regressions as it has been shown to generate stronger results than other group level measures \citep{burke2013classroom}. In total we have X observations and X variables. We use two different types of data in this study. The first type is classroom level data, that consists of individual student level characteristics as well as classroom characteristics. We use data from the first class taken by first year students in order to mitigate selection bias.\footnote{Discussed in the Empirical Methodology section.} The second type of data we use is point data, which consists of the number of points student bid on classes and their preferences for a certain class. 

\subsection{Classroom Data}

%We choose to use only data on the first class taken by first year students in order to mitigate selection bias. First year students must select their first course before the semester begins using a point bidding system. Students are accepted into the institution from a variety of areas and backgrounds, and it is very unlikely that students were able to select courses based on any previous relationships. This reduces the likelihood of selection bias effecting our model. In addition, first year students choose courses for a variety of reasons, and these reasons add an element of randomness to the course selection, which also helps to mitigate selection bias. 

The classroom level data on first year students in their first class at the institution is primarily used in the second stage (primary) regression.\footnote{Some classroom level data was also used in the first stage regression. See Point Data for details.}
There are five key variables in our primary regression, Grade, AcadRating (Academic Rating), PctTopX, PctBotX, and PctMidY. 
Grade is the primary outcome of interest, and is the grade received by a student after taking the course. 
Courses are graded on a four point scale, and there are eleven possible grades ranging from an ``A'' (4.0) to ``F'' (0.0).\footnote{The possible grades are A = 4.0, A- = 3.7, B+ = 3.3, B = 3.0, B- = 2.7, C+ = 2.3, C = 2.0, C- = 1.7, D+ = 1.3, D = 1.0, F = 0.0} 

As a measure for student ability we follow the literature and use a proxy measure developed on data prior to college enrollment, Academic Rating \citep{griffith2014peer}. 
The Academic Rating is a number assigned to all students at the institution. 
It is a number that represents the culmination of a student's high school GPA, their test scores, the difficulty of the high school curriculum, the quality of their high school, and their writing ability. 
As suggested by the literature these variables are all common (and ``good'') indicators of college academic performance \citep{betts2003determinants,dooley2012persistence}.
In our data the Academic Rating variable ranges from a low of 18 (representing a low ability student) and a high of 65 (representing a high ability student).

We used the Academic Rating variable to create our peer measure variables PctTopX, PctBotX, and PctMidY. PctTopX represents the proportion of students in the class the top X percent of the sample based on Academic Rating. 
For instance, PctTop5 represents the proportion of students in the class that are in the top five percent of the sample based on Academic Rating. 
PctBotX is defined similarly for the bottom X percentage in terms of Academic Rating. 
PctMidY is the proportion of students in a class that are not in the top X percent or bottom X percent of the sample in terms of Academic Rating. 
The remaining variables, Minority, Female, InState, International, Needy, ClassSize, URM, Year, Division, and Professor, were used as control variables in the regression.
Refer to Table 3.1 below for the definitions of all the class level variables used. 

\clearpage{}

\begin{table}[htb]
\centering
\caption{Classroom Variable Definitions}
 \begin{tabular}{||p{0.15\linewidth}|p{0.8\linewidth}||} 
 \hline
 Variable & Definition \\ [0.5ex] 
 \hline\hline
 Grade & The grade recieved by a student after taking the course. \\ 
 \hline
 AcadRating & Reffered to as Academic Rating, a number that represents the culmination of a student's high school GPA, their test scores, the difficulty of the high school curriculum, the quality of their high school, and their writing ability. \\
 \hline
 PctTopX & The proportion of students in the class in the top X percent of the sample based on Academic Rating \\
 \hline
 PctBotX & The proportion of students in the class in the bottom X percent of the sample based on Academic Rating \\
 \hline
 PctMidY & The proportion of students in a class that are not in the top X percent or bottom X percent of the sample in terms of Academic Rating. \\
 \hline
 Minority & A dummy variable representing whether or not the student is non-caucasian. 1 = non-caucasian \& 0 = caucasian\\
 \hline
 Female & A dummy variable representing whether or not the student identifies as a female. 1 = female \& 0 = male\\
 \hline
 InState & A dummy variable inidicating whether or not the student is an in-state student.  1 = in-state \& 0 = out of state\\
 \hline
 International & A dummy variable inidicating whether or not the student is an international student.  1 = international \& 0 = not international\\
 \hline
%TODO: Check this
 Needy & Whether or not the student was offered financial aid.  1 = financial aid \& 0 = no financial aid\\
 \hline
 ClassSize & An integer the represents the total number of students in a class. \\
 \hline
 URM & Under Represented Minority, a dummy variable inidicating whether or not the student is non-caucaian or Asian.  1 = Asian or Caucasian \& 0 = other ethnicity \\
 \hline
 Year & The year the class took place. \\
 \hline
 Division & The subject area of the class, either Natural Science, Social Sciences, or Humanities. \\
 \hline
 Professor & An identifier for the professor teaching the class. \\
[1ex] 
 \hline
\end{tabular}
\end{table}

\clearpage{}

\subsection{Points Data}

At this institution a bidding system is used to ration classes. 
Students are allotted twenty points and must rank eight classes in terms of their preferences. 
After classes are ranked students then must bid a number between zero and twenty (inclusive) points per class on their list. 
Students with the highest number of points bid per class are allotted seats, and ties are broken randomly. 
If a student does not make it into any class on his or her preference list, then a class is chosen for the student at random. 

The key variables in our first stage regression are Ranking and Demand. 
The dependent variable in the first stage regression is Ranking, a number between one and eight specifying the student's preference for the course, where one is a high preference and eight is a low preference. 
The Demand variable represents the total number of points bid on a course divided by the number of bidders. 
It is an independent variable we created specifically for the first stage regression that has no effect on the second stage regression. 
We used the remaining variables in the first stage regression, Minority, Female, InState, International, Needy, AcadRating, URM, and Subject as control variables. Refer to Table 3.2 below for the definitions of the unique variables used in the first stage regression. 

\clearpage{}

\begin{table}[htb]
  \centering
  \caption{Points Data Variable Definitions}
  \begin{tabular}{||p{0.15\linewidth}|p{0.8\linewidth}||} 
    \hline
    Variable & Definition \\ [0.5ex] 
    \hline\hline
    Ranking & A number between one and eight specifying the student's preference for the course, where one is a high preference and eight is a low preference. \\ 
    \hline
    Demand & The total number of points bid on a course divided by the number of bidders. \\
    \hline
    % TODO: Create full list
    Subject & The specific subject of the course, such as mathematics, anthropology, chemistry, psychology, etc. For a full list of subjects see Appendix A. \\
    [1ex] 
    \hline
  \end{tabular}
\end{table}

\subsection{Summary Statistics}

The descriptive statistics for the non-dummy variables are given below in Table 2.1.\footnote{Control variables, such as Year, Professor, Division, and Subject where not summarized.} The outcome of interest, Grade, has a mean of 3.14 on a 4 point scale and a standard deviation of 0.855. The student ability measure (Academic Rating) ranges from a minimum of 18 to a maximum of 65, and has a mean of 51.12. On average 4.68 points were bid on a class, with the minimum demand being 0.84 points and the maximum being 14.12 points. PctTop5 has a minimum value of 0 and maximum value of 0.44, while PctBot5 has a minimum value of 0 and maximum value of 0.53. Finally, our data set has an average class size of 12 with a standard deviation of 3. 

It may be valuable to note that several studies suggest that smaller class sizes have a positive impact on average student achievement (measured by grades and test scores) \citep{diette2015class,kokkelenberg2008effects}. The primary reasoning for the inverse relationship between achievement and class size is that students have more quality time to interact with teachers and peers as class size decreases. As this institution has relatively small class sizes, we may expect to find magnified peer effects. 

\newpage{}


\bibliography{references}

\end{document}
